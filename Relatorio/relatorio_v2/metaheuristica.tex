%%%%%%%%%%%%%%%%%%%%%%%%%%%%%%%%%%%%%%%%%%%%%%%%%%%%%%%%%%%%%%%%%%%%%%%%%%%%%%%%%%%%%%%%%%%%%%%%
%                                                                                              %
%                             Definicao para a classe Artigo                                   %
%                                                                                              %
%%%%%%%%%%%%%%%%%%%%%%%%%%%%%%%%%%%%%%%%%%%%%%%%%%%%%%%%%%%%%%%%%%%%%%%%%%%%%%%%%%%%%%%%%%%%%%%%

\documentclass[portugues, brazil, a4paper,12pt]{article}
\bibliographystyle{plain}

%%%%%%%%%%%%%%%%%%%%%%%%%%%%%%%%%%%%%%%%%%%%%%%%%%%%%%%%%%%%%%%%%%%%%%%%%%%%%%%%%%%%%%%%%%%%%%%%
%                                                                                              %
%                       Pacotes a utilizar na compilacao do documento                          %
%                                                                                              %
%%%%%%%%%%%%%%%%%%%%%%%%%%%%%%%%%%%%%%%%%%%%%%%%%%%%%%%%%%%%%%%%%%%%%%%%%%%%%%%%%%%%%%%%%%%%%%%%

\usepackage[brazil]{babel}
\usepackage{graphicx}
\usepackage{geometry}
\usepackage[utf8]{inputenc}
\usepackage[T1]{fontenc}
\usepackage{epstopdf}
\usepackage{hyperref}
\usepackage{float}
\usepackage{gensymb}
\usepackage{color}
\usepackage{url}

\usepackage{graphicx,url}

\usepackage[brazil]{babel}
\usepackage{verbatim}
\usepackage{todonotes}
\usepackage{longtable}
\usepackage{amssymb}
\usepackage{amsmath}
\usepackage{float}
\usepackage{amsfonts}
\usepackage{algorithm, algpseudocode}
\usepackage{color}
\usepackage{minted}
\usepackage{url}
\usepackage{multicol}
%\usepackage[latin1]{inputenc}
\usepackage[utf8]{inputenc}
\usepackage{longtable}

\hypersetup{
    colorlinks,
    citecolor=black,
    filecolor=black,
    linkcolor=black,
    urlcolor=black
}



\makeatletter
\renewcommand{\paragraph}{\@startsection{paragraph}{4}{0ex}%
   {-3.25ex plus -1ex minus -0.2ex}%
   {1.5ex plus 0.2ex}%
   {\normalfont\normalsize\bfseries}}
\makeatother

\stepcounter{secnumdepth}
\stepcounter{tocdepth}

%%%%%%%%%%%%%%%%%%%%%%%%%%%%%%%%%%%%%%%%%%%%%%%%%%%%%%%%%%%%%%%%%%%%%%%%%%%%%%%%%%%%%%%%%%%%%%%%
%                                                                                              %
%                       Configuracao dos pacotes utilizados no doc.                            %
%                                                                                              %
%%%%%%%%%%%%%%%%%%%%%%%%%%%%%%%%%%%%%%%%%%%%%%%%%%%%%%%%%%%%%%%%%%%%%%%%%%%%%%%%%%%%%%%%%%%%%%%%

\geometry{a4paper,left=3cm,right=3cm,top=2.5cm,bottom=2.93cm}


%%%%%%%%%%%%%%%%%%%%%%%%%%%%%%%%%%%%%%%%%%%%%%%%%%%%%%%%%%%%%%%%%%%%%%%%%%%%%%%%%%%%%%%%%%%%%%%%
%                                                                                              %
%                                   Capa do Documento                                          %
%                                                                                              %
%%%%%%%%%%%%%%%%%%%%%%%%%%%%%%%%%%%%%%%%%%%%%%%%%%%%%%%%%%%%%%%%%%%%%%%%%%%%%%%%%%%%%%%%%%%%%%%%

\begin{document}

\begin{titlepage}

  \vfill

	\begin{figure}[H]
	\centering
		\includegraphics[scale=0.15]{img/logo-ufop.jpg}
	\end{figure}

  \vfill

  \begin{center}
    \begin{Large}
      \textbf{UNIVERSIDADE FEDERAL DE OURO PRETO}
    \end{Large}
  \end{center}

  \begin{center}
    \begin{large}
      \textbf{Mestrado em Ciência da Computação} \\[1.4cm] 
    \end{large}
  \end{center}

  \vfill

  \begin{center}
    \begin{large}
      \textbf{Projeto e Análise de Algoritmos -- \\ Problema Problema da Alocação Generalizada (GAP) com uso de Algoritmos Heurísticos} \\[0.4cm] 
    \end{large}
  \end{center}

  \vfill

  \begin{center}
    \begin{large}
      Autor: \\
		Rodolfo Labiapari Mansur Guimarães - rodolfolabiapari@gmail.com
    \end{large}
  \end{center}

	\vfill

  \begin{center}
    \begin{large}
      Professor Orientador: \\
      Haroldo Gambini Santos - haroldo@iceb.ufop.br
    \end{large}
  \end{center}

  \vfill

  \begin{center}
    \begin{large}
      Ouro Preto - MG \\
      \today \\
    \end{large}
  \end{center}

\clearpage
\tableofcontents 
\end{titlepage}

%%%%%%%%%%%%%%%%%%%%%%%%%%%%%%%%%%%%%%%%%%%%%%%%%%%%%%%%%%%%%%%%%%%%%%%%%%%%%%%%%%%%%%%%%%%%%%%%
%                                                                                              %
%                               Introducao ao trabalho                                         %
%                                                                                              %
%%%%%%%%%%%%%%%%%%%%%%%%%%%%%%%%%%%%%%%%%%%%%%%%%%%%%%%%%%%%%%%%%%%%%%%%%%%%%%%%%%%%%%%%%%%%%%%%

\section{Introdução}
	\subsection{Problema da Alocação Generalizada}
		Neste capítulo, faz-se descrições do Problema de Alocação Generalizada (do inglês, \textit{Generalizated Assingment Problem}, GAP) em sua forma clássica.
	
		GAP é um problema de alocação de recursos, denominado tarefas, para determinados agentes, alocadores, com propósito de custo mínimo.

		$$\min \sum_{i \in I}^{} \sum_{j \in J}^{} c_{ij}x_{ij}$$

		Os elementos básicos da fórmula pra este problema são:

		\begin{itemize}
			\item Um conjunto $I$ de agentes $(i = 1, 2, 3, \ldots, m)$;
			
			\item Um conjunto $J$ de agentes $(j = 1, 2, 3, \ldots, n)$;
		\end{itemize}

		Cada tarefa $T_j \in T$ consome uma quantidade de recursos $a_{ij}$ do agente $i \in I$, ou seja, consome uma parte da capacidade do agente, a um diferente custo $c_{ij}$.
		
		Este problema deve atender a três restrições:
		
		\begin{enumerate}
			\item Cada agente tem uma capacidade limitada;
			
			\item Cada tarefa sé pode ser alocada a um único agente;
			
			\item Todas as tarefas devem ser alocadas;
		\end{enumerate}
		
		A solução para o GAP é um vetor de $n$ elementos, onde a $k$-ésima posição do vetor guarda o agente ao qual a $k$-ésima tarefa foi associada.
		
		Dasgupta explica \cite{Dasgupta2008}:
		
		\begin{quotation}
			{\it ``Existe um número de agentes e um número de tarefas. Podemos alocar qualquer agente a qualquer uma das tarefas, porém, cada alocação tem um custo que pode variar dependendo da tarefa e agente específicos. É necessário que todas as tarefas sejam feitas, designando exatamente um agente para cada tarefa de modo que o custo total (a soma) de todas as alocações seja minimizado.''}
		\end{quotation}
		
		
	\subsection{Heurística em Algoritmos}
		As heurísticas (ou modelos heurísticos) não garantem que seja encontrado o ótimo. Geralmente encontram soluções próximas à ótima, mas algumas vezes, dependendo das circunstâncias, é possível que sejam obtidos resultados arbitrariamente ruins (ou também o resultado ótimo). Isso denota uma relação de custo benefício. Ao utilizarmos este tipo de método, não precisamos da melhor solução possível, mas geralmente obtemos uma solução viável com um tempo computacional aceitável.
		
		Dentro da categoria das heurísticas estão alguns métodos caracterizados como metaheurísticas. São definidos como uma metaheurística os métodos que otimizam um problema ao iterativamente buscar a melhora de uma canditada à solução, utilizando alguma medida de qualidade. São métodos de otimização estocástica, ou seja, algoritmos e técnicas que utilizam alguma forma de aleatoriedade para encontrar uma solução ótima ou próxima à ótima em problemas difíceis de serem resolvidos.
		
		Metaheurísticas são utilizadas para encontrar soluções para problemas para os quais não se tem muitas informações. Não se sabe qual deve ser a solução ótima e o espaço de busca é vasto. Mas, se existe uma candidata à solução do problema, é possível testá-la e definir o quão boa ela é. Isto é verdade para muitos problemas de Otimização Combinatória, o que leva uma metaheurística a ser bastante utilizada para resolver problemas da área.


%%%%%%%%%%%%%%%%%%%%%%%%%%%%%%%%%%%%%%%%%%%%%%%%%%%%%%%%%%%%%%%%%%%%%%%%%%%%%%%%%%%%%%%%%%%%%%%%
%                                                                                              %
%                               Introducao ao trabalho                                         %
%                                                                                              %
%%%%%%%%%%%%%%%%%%%%%%%%%%%%%%%%%%%%%%%%%%%%%%%%%%%%%%%%%%%%%%%%%%%%%%%%%%%%%%%%%%%%%%%%%%%%%%%%

\section{Decisões Iniciais de Projeto}
	\subsection{Modularização de Funções em Arquivos}
		Primeiramente, como se trata de um único problema abordado em vários algoritmos, percebeu-se que poderia desenvolver uma única estrutura de dados para para todos os tipos de algoritmos a serem implementados além de funções genéricas do problema que seriam comuns entre alguns aloritoms. Com isso, decidiu-se inicialmente serparar a estrutura de dados e algumas funções comumente utilizadas em um arquivo de funções incluído à cada algoritmo.

		A modularização deste trabalho trouxe várias melhorias como portabilidade do problema, além de boa refatoração do código de modo que, realizando uma alteração no arquivo de funções, todos os algoritmos serão alterados automaticamente economizando tempo e reduzindo a quantidade de linhas de código por algoritmo.

	\subsection{Estrutura de Dados} \label{sec:ed}
		Utilizou-se de uma única estrutura de dados que representaria o problema no qual permitiu-se ser aplicada em todos os quartos algoritmos implementados situada no arquivo de cabeçalho. Como é uma estrutura geral, será descrita previamente.

		A estrutura segue abaixo:

		\begin{minted} [frame=lines, framesep=2mm, tabsize=3, breaklines=true, baselinestretch=1.2, linenos, fontsize=\footnotesize ]{c}
typedef struct Struct_Solucao {
	int * excesso;
	int * tarefas;
	double avaliacao;
	double custo;
} Solucao;
		\end{minted}

		A estrutura é formada por duas variáveis e dois vetores que representam as informações de cada solução. O vetor \textit{tarefas} armazena o agente responsável por realizar a tarefa de índice $i$ e consequentemente, o vetor excesso armazena a quantidade de recursos utilizada pelo agente $i$. Este vetor de excessos é comparado com o vetor de capacidade de cada agente verificando se alguma posição ultrapassou a quantidade máxima de recursos que determinado agente pode exercer.

		Os dois vetores são necessários para calcular as duas outras variáveis que são avaliação e custo. Custo é a soma de todos os custos de cada agente em cada tarefa e avaliação é o quão boa é esta solução em relação à suas características. A avaliação destes serão descritos no decorrer do documento.


	\subsection{Geração de Vizinhos Otimizada} \label{sec:otimizacao}
		Após entender como funciona a estrutura de dados do problema, deve-se entender como é feita a geração de vizinhos já que este procedimento é utilizado em três dos quatros algoritmos implementados e todos eles utilizam o mesmo procedimento sem alterações.

		Deve-se atentar ao detalhe que esta função é otimizada ao ponto de avaliar a solução gerada sem invocar o procedimento \verb|Avalia_Solucao(Solucao)|. O procedimento é descrito à seguir.

		\begin{minted} [frame=lines, framesep=2mm, tabsize=3, breaklines=true, baselinestretch=1.2, linenos, fontsize=\footnotesize ]{c}
void Gera_Vizinho(Solucao * atual, Solucao ** proxima) {
	int i = 0, j = 0, agente_atual = 0, agente_novo = 0, 
	tarefa_escolhida1 = 0, sum_recursos = 0, 
	quant_alteracoes = 0;
	char solucao_invalida = 0;
	
	// Copia os valores do atual
	for (i = 0; i < QUANT_TAREFAS; i++) {
		(*proxima)->tarefas[i] = atual->tarefas[i];
		if (i < QUANT_AGENTES)
			(*proxima)->excesso[i] = atual->excesso[i];
	}

	(*proxima)->custo = atual->custo;

	
	// Define uma quantidade de alterações pra gerar o vizinho
	quant_alteracoes = 2;
	
	// Altera o indivíduo
	for (i = 0; i < quant_alteracoes; i++) {

		// Escolhe a tarefa que será alterada
		tarefa_escolhida1 = random() % QUANT_TAREFAS;
		agente_atual = (*proxima)->tarefas[tarefa_escolhida1]; 

		// Gera um novo agente pra ela e certifica que ele é diferente do anterior.
		do {
			agente_novo = random() % QUANT_AGENTES;
		} while (agente_novo == agente_atual);

		// Atribui o novo agente à tarefa
		(*proxima)->tarefas[tarefa_escolhida1] = agente_novo;

		// Procedimento Otimizado:
			// A cada alteração, realiza-se a alteração dos valores da nova geração.
			// Como este novo vizinho não é feito do zero e sim sobre cópia de um anterior,
			// basta alterar os valores herdados do seu anterior, atualiando em O(1)

		// Sendo assim, atualiza o excesso de cada agente
			// Retira recurso do agente que ficou livre
		(*proxima)->excesso[agente_atual] -= RECURSOS_A_T[agente_atual][tarefa_escolhida1];
			// Acrescenta recurso do agente que recebeu a tarefa atual
		(*proxima)->excesso[agente_novo ] += RECURSOS_A_T[agente_novo ][tarefa_escolhida1]; 

		// Calcula o custo atual desta solução
		(*proxima)->custo += CUSTO_A_T[agente_novo ][tarefa_escolhida1] - CUSTO_A_T[agente_atual][tarefa_escolhida1];
	}

	// Verifica se a solução gerada é válida
	for (j = 0; j < QUANT_AGENTES; j++) {
		sum_recursos += (*proxima)->excesso[j];

		if ((*proxima)->excesso[j] > CAPAC_AGENTES[j]) {
			solucao_invalida = 1;
		}
	}

	// Calcula o fator avaliação 
	if (solucao_invalida) 
		(*proxima)->avaliacao = ((double) sum_recursos ) * 1000000;
	else {
		(*proxima)->avaliacao = ((double)  (*proxima)->custo);
	}
}
		\end{minted}

		A geração é um procedimento simples que realiza a alteração de apenas dois itens dentro da solução gerando uma nova solução com valores diferentes da original e sem afastar do local do espaço de solução. Assim, primeiro é copiado todos os dados da solução original e depois realizado duas alterações em qualquer uma das tarefas da solução alterando o agente delas. 

		Recapitulando, a cada alteração da tarefa todos suas propriedades são recalculadas em tempo constante ao contrário da invocação do método \verb|Avalia_Solucao(Solucao)| que executaria em O(n). A vantagem de executar o método de forma mais veloz proporciona que o algoritmo ao todo possa executar mais tarefas sobre o mesmo intervalo de tempo e assim, aumentando a probabilidade de encontrar uma solução melhor ainda.

		A forma de avaliação propriamente dita é descrita a seguir.


	\subsection{Função Avaliação}
		\begin{minted} [frame=lines, framesep=2mm, tabsize=3, breaklines=true, baselinestretch=1.2, linenos, fontsize=\footnotesize ]{c}
double Avalia_Solucao(Solucao * sol) {
	int i = 0, capacidade_agentes[QUANT_AGENTES], sum_recursos = 0;
	double custo = 0; char solucao_invalida = 0;
	
	// Define a capacidade inicial utilizada de cada agente com 0
	for (i = 0; i < QUANT_AGENTES; i++)
		capacidade_agentes[i] = 0;

	// Realiza os cálculos de custo e capacidade
	for (i = 0; i < QUANT_TAREFAS; i++) {
		custo += CUSTO_A_T[sol->tarefas[i]][i];
		capacidade_agentes[sol->tarefas[i]] += RECURSOS_A_T[sol->tarefas[i]][i];
	}

	// Verifica se algum agente passou sua capacidade máxima
	for (i = 0; i < QUANT_AGENTES; i++) {
		sol->excesso[i] = capacidade_agentes[i];
		sum_recursos += capacidade_agentes[i];
	
		// Se sim define esta solução como inválida
		if (capacidade_agentes[i] > CAPAC_AGENTES[i])
			solucao_invalida = 1;
	}
	
	sol->custo = custo;
	
	// Caso a solução foi excedida, altera a avaliação do indivíduo tornando-o
		// pior.
	if (solucao_invalida) 
		sol->avaliacao = ((double) sum_recursos ) * 1000000;
	else {
		sol->avaliacao = ((double) custo);
	}
	
	return sol->avaliacao;
}
		\end{minted}

		O procedimento inicia calculando a quantidade de recursos utilizado por cada agente além do custo total da solução independente dela ser uma solução válida ou não.

		Após calculado, é feito uma verificação (linha 16-23) onde analisa se pelo menos um agente ultrapassou sua quantidade de recursos limite, tornando a solução inválida. Após analisado, é realizado duas tomadas de decisão já suponto o problema de minimização, sendo elas quando a solução for:

		\begin{itemize}
			\item \textbf{Inválida:} O valor da avalização será calculado de forma diferente da solução válida. Foi desenvolvido um novo método para conseguir gerar soluções iniciais válidas de forma mais rápida no algoritmo a fim de permitir que ele trabalhe na maior parte do tempo com soluções válidas do que simplesmente procurando soluções. A fórmula desenvolvida para tal é $quantidade\_recurso\_utilizado * 1000000$. 

			É possível perceber que o fator custo não entra neste cálculo. Isso deve ao fato de soluções inválidas serem dependente somente do fator de \textit{recurso} já que, diferente do fator \textit{custo} no qual queremos otimizar, o fator \textit{recurso} é uma restrição. Se esta restrição não for satisfeita, então todo o resto é inválido, justificando assim o uso desta fórmula de avalização. 

			Por fim, existe a penalização de soluções inválidas. Como o procedimento é de minimização, multiplicou-se esse valor por um milhão tornando esta solução totalmente inválida;


			\item \textbf{Válida:} O valor da avaliação será obtido pela fórmula $custo$. Uma vez obtido uma solução válida (que satisfaz todas as restrições), não queremos minimizar as restrições mas sim o seu fator \textit{custo}. Sendo assim, a fórmula de avaliação de soluções válidas é somente o valor de seu custo válido, com intuito da minimização ser mais clara e objetiva possível.
		\end{itemize}



%%%%%%%%%%%%%%%%%%%%%%%%%%%%%%%%%%%%%%%%%%%%%%%%%%%%%%%%%%%%%%%%%%%%%%%%%%%%%%%%%%%%%%%%%%%%%%%%
%                                                                                              %
%                                   Algoritmo Genético                                         %
%                                                                                              %
%%%%%%%%%%%%%%%%%%%%%%%%%%%%%%%%%%%%%%%%%%%%%%%%%%%%%%%%%%%%%%%%%%%%%%%%%%%%%%%%%%%%%%%%%%%%%%%%

\section{Algoritmo Genético}

	\subsection{Execução}
		O algoritmo implementado necessita somente de duas populações instanciadas pra realizar suas operações. Isso pois elas comutação entre si na realização do papel principal de \textit{população atual} sendo uma servindo de \textit{buffer} da outra a cada iteração.

		Assim, é instanciadas duas populações. Os novos filhos gerados e mutados serão postos na segunda população e em seguida será completada com indivíduos da primeira de forma aleatória. Esta estratégia fio desenvolvida para reduzir chamadas de sistema para alocação e liberação de memória.
		
		Abaixo serão descrito alguns procedimentos fundamentais do Algoritmo Genético.

		\subsubsection{Criando Uma População Inicial}
			O processo de geração de população inicial funciona de duas formas. A primeira utiliza escolha aleatória de agentes e atribuindo a tarefa $i$. Já a segunda é uma forma gulosa de preenchimento de tarefas. Para a tarefa $i$ é vasculhado todos os agentes procurando o que possui o que gasta o menor recurso pra executar esta tarefa e assim o é escolhido.

			O primeiro indivíduo escolhido é gerado totalmente pelo procedimento guloso. Os demais são uma mescla de guloso com aleatoriedade onde o algoritmo aleatório possui 66\% de chance de ser escolhido para determinar a tarefa $i$, restando 33\% para o guloso.

			O algoritmo é descrito a seguir.

		\begin{minted} [frame=lines, framesep=2mm, tabsize=3, breaklines=true, baselinestretch=1.2, linenos, fontsize=\footnotesize ]{c}
void Cria_Nova_Populacao(Individuo *** P) {
	int i = 0, j = 0, k = 0, menor = 0;
	Individuo ** p_local = 0;

	p_local = *P;

	// Para cada item a ser criado
	for (i = 0; i < TAM_POP; i++) {
		
		// Aloca suas variáveis que armazenarão suas informações
		p_local[i]          = calloc(1, sizeof(Individuo));
		p_local[i]->excesso = calloc(QUANT_AGENTES, sizeof(int));
		p_local[i]->tarefas = calloc(QUANT_TAREFAS, sizeof(int));
		
		// Gera valores pra este indivíduo
		for (j = 0; j < QUANT_TAREFAS; j++) {
			// O primeiro indivíduo será gerado de forma gulosa e os outros
				// Serão uma mistura de Guloso com Aleatoriedade
			
			// Se não for o primeiro indivíduo, possui 66% de gerar valores
				// por meio de função randomica
			if (i > 0 && random() % 3 != 0) {
				p_local[i]->tarefas[j] = random() % QUANT_AGENTES;
				
			// Caso contrário, utiliza uma geração gulosa pra esta tarefa.
			} else {
				// O método guloso escolhe o recurso mais leve desta tarefa
				p_local[i]->tarefas[j] = 0;
				menor = 0;

				// Seleciona o agente que utiliza o menor recurso desta
					// tarefa
				for (k = 1; k < QUANT_AGENTES; k++) {
					if (RECURSOS_A_T[k][j] < RECURSOS_A_T[menor][j]) {
						menor = k;
						p_local[i]->tarefas[j] = k;
					}
				}
			}
		}

		// Avalia o novo indivíduo gerado
		Avalia_Individuo(p_local[i]);
	}
		\end{minted}
		

		\subsubsection{Geração de Novos Filhos e Mutação}
			A geração de filhos é dividida em duas partes sendo a primeira a geração de filhos derivando de dois pais distintos e a segunda sua mutação.
			
			A geração de novos filhos acontece selecionando dois pais por torneio e realizando o cruzamento uniforme entre eles. A quantidade de filhos gerados é calculada pela fórmula $TAM\_POP * TAX\_CRUZAM + 1$ sendo a taxa de cruzamento a porcentagem em relação à população, e o valor $ + 1$ representando pelo menos 1 cruzamento por iteração. Os novos filhos gerados (ainda não mutados) serão salvos na futura população para que o processamento possa continuar.

			Gerado os filhos, inicia-se o processo de mutação. Como a próxima população só contem indivíduos gerados nesta iteração, é realizado a mutação de todos os eles. A quantidade de mutação que um indivíduo receberá é calculado por $QUANT\_TAREFAS * TAX\_MUT$, sendo a taxa de mutação a porcentagem da quantidade de tarefas do indivíduo. A mutação é feita alterando os agentes aleatoriamente das tarefas escolhidas ao acaso. Ao final é feito a avaliação do indivíduo gerado.
			

		\subsubsection{Seleção}
			Ao final destes processos descritos, a nova população possuirá apenas filhos novos mutados e por isso deve ser preenchida com indivíduos aleatórios da população anterior. 

			Este processo acontece com dois passos. O primeiro é escolhendo o indivíduo com melhor avaliação (elite) e adicionando-o na nova população. Adicionado este indivíduo, realiza-se o preenchimento dela com indivíduos aleatórios.

			Ao final, terá uma população pronta para iterar novamente no algoritmo. 
	
	
	\subsection{Arquivos de Configuração de Execução}
		O arquivo de configuração deste algoritmo é composto dos seguintes itens:
		
		\begin{enumerate}
			\item \textbf{Quantidade de Indivíduos na População:} Quantidade de possíveis soluções que representarão uma População no algoritmo; e
			
			\item \textbf{Valor da Taxa de Geração de Novos Filhos em Relação à População:} Valor real representando qual a porcentagem de filhos a serem gerados em relação ao tamanho da população; e
			
			\item \textbf{Valor da Taxa de Mutação em Relação ao Tamanho da Solução a ser Mutada:} Valor real que representa qual a porcentagem de mutação que determinado filho receberá ao compor à nova População.
			
			\item \textbf{Tempo em Segundos}.
		\end{enumerate}
		
		A configuração utilizada no problema foi:
		
		\begin{verbatim}
		10000
		0.2
		0.01
		60
		\end{verbatim}



%%%%%%%%%%%%%%%%%%%%%%%%%%%%%%%%%%%%%%%%%%%%%%%%%%%%%%%%%%%%%%%%%%%%%%%%%%%%%%%%%%%%%%%%%%%%%%%%
%                                                                                              %
%                                  Recozimento Simulado                                        %
%                                                                                              %
%%%%%%%%%%%%%%%%%%%%%%%%%%%%%%%%%%%%%%%%%%%%%%%%%%%%%%%%%%%%%%%%%%%%%%%%%%%%%%%%%%%%%%%%%%%%%%%%

\section{Algoritmo de Recozimento Simulado}
	Aqui exibido detalhes do Algoritmo de Recozimento Simulado (do inglês \textit{Simulated Annealing}, SA) implementado para este problema.
	
	O algoritmo implementado foi baseado no livro sobre Meta-heurísticas \cite{Lopes2013}, disponibilizado gratuitamente ao público pela editora Omnipax.
	
	\subsection{Execução}

		\subsubsection{Atualização de Temperatura}
	
			Utilizou-se de um método que utiliza cálculo geométrico para a atualização da temperatura. Assim, a temperatura é iniciada com um valor definido pelo arquivo de configuração do algoritmo e este valor é decaído até chegar em uma temperatura igual à 0,2.
	
			\begin{minted} [frame=lines, framesep=2mm, tabsize=3, breaklines=true, baselinestretch=1.2, linenos, fontsize=\footnotesize ]{c}
void Atualiza_Temperatura(double * t) {
	*t = 0.995 * *t;
}
			\end{minted}
			
			Com uma temperatura decaindo de forma lenta, é possível percorrer mais busca locais a procura de soluções vizinhas melhores. Da mesma forma, o fator \textit{temperatura} decairá de forma lenta quando estiver próximo do valor 1, aperfeiçoando ainda mais a solução encontrada.
	
	\subsection{Arquivos de Configuração de Execução}
		O arquivo de configuração deste algoritmo é composto dos seguintes itens:
		
		\begin{enumerate}
			\item \textbf{Valor Inteiro da Temperatura Inicial:} Configuração do valor inicial da temperatura ao iniciar o processamento do problema; e
			
			\item \textbf{Quantidade de Iterações:} Número de iterações de busca realizados em cada alteração da temperatura. Este valor representa uma busca em soluções melhores a fim de aprimorar a solução da temperatura atual.
		\end{enumerate}
		
		A configuração utilizada no problema foi:
		
		\begin{verbatim}
		50
		200000
		\end{verbatim}
	


%%%%%%%%%%%%%%%%%%%%%%%%%%%%%%%%%%%%%%%%%%%%%%%%%%%%%%%%%%%%%%%%%%%%%%%%%%%%%%%%%%%%%%%%%%%%%%%%
%                                                                                              %
%                                           Greedy                                             %
%                                                                                              %
%%%%%%%%%%%%%%%%%%%%%%%%%%%%%%%%%%%%%%%%%%%%%%%%%%%%%%%%%%%%%%%%%%%%%%%%%%%%%%%%%%%%%%%%%%%%%%%%

\section{Método Reinício}
	Aqui exibido detalhes do Método Reinício implementado para este problema.
	
	O algoritmo implementado foi baseado no pseudocódigo disponibilizado pelo professor-orientador da disciplina.
	
	\subsection{Execução}
		O algoritmo baseia-se na simples ideia de gerar uma solução inicial e realizar $i$ iterações sobre ela a procura de vizinhos que possuam uma avaliação melhor. Isso repete enquanto houver tempo necessário para processamento.
	
	
	\subsection{Arquivos de Configuração de Execução}
		O arquivo de configuração deste algoritmo é composto dos seguintes itens:
		
		\begin{enumerate}
			\item \textbf{Valor Inteiro de Tempo de Processamento:} Quantidade de segundos de processamento; e
			
			\item \textbf{Quantidade de Iterações:} Número de iterações de busca realizados na solução gerada.
		\end{enumerate}
		
		A configuração utilizada no problema foi:
		
		\begin{verbatim}
		60
		10000000
		\end{verbatim}
	


%%%%%%%%%%%%%%%%%%%%%%%%%%%%%%%%%%%%%%%%%%%%%%%%%%%%%%%%%%%%%%%%%%%%%%%%%%%%%%%%%%%%%%%%%%%%%%%%
%                                                                                              %
%                                             GRASP                                            %
%                                                                                              %
%%%%%%%%%%%%%%%%%%%%%%%%%%%%%%%%%%%%%%%%%%%%%%%%%%%%%%%%%%%%%%%%%%%%%%%%%%%%%%%%%%%%%%%%%%%%%%%%

\section{Greedy Randomized Adaptive Search Procedure (GRASP)}
	Utilizou-se também uma versão do algotitmo GRASP para a procura de soluções boas para o GAP.
	
	\subsection{Execução}
	
		\subsubsection{Construção Randomicamente Gulosa com Otimização em Avalização de Solução}
			Para este procedimento de geração de solução inicial, definiu-se que a lista RCL os possíveis agentes $j$ para a tarefa $i$.

			Assim, como é possível ver no algoritmo abaixo, a lista RCL compões de todos os agentes para a tarefa $i$. Entretanto, é realizado o cálculo do fator $ fator = agente_{min} + \alpha (agente_{mx} - agente_{min}) $, onde $agente_{min}$ e $agente_{max}$ são obitidos pelo menor e maior valor de $recurso + custo$ dos agentes $j$ naquela tarefa $i$, respectivamente, e $\alpha$ é a porcentagem de quão guloso ou aleatório o algoritmo irá se comportar.

			Ao final, o agente escolhido para a tarefa $i$ é o agente $j$ com valor mais próximo ao fator calculado.
	
			\begin{minted} [frame=lines, framesep=2mm, tabsize=3, breaklines=true, baselinestretch=1.2, linenos, fontsize=\footnotesize ]{c}
void GreedyRandomizedConstruction(Solucao ** s, float alfa) {
	int i = 0, j = 0, min = 0, max = 0, fator = 0, sum_recursos = 0;
	char solucao_invalida = 0;

	for (i = 0; i < QUANT_AGENTES; i++)
		(*s)->excesso[i] = 0;

	(*s)->custo = 0;
	
	// Para cada tarefa
	for (i = 0; i < QUANT_TAREFAS; i++) {
		min = max = 0;

		// Encontra os valores máximos e mínimos dos agentes
		for (j = 1; j < QUANT_AGENTES; j++) {
			if (RECURSOS_A_T[j][i] + CUSTO_A_T[j][i] < RECURSOS_A_T[min][i] + CUSTO_A_T[min][i])
				min = j;
			
			if (RECURSOS_A_T[j][i] + CUSTO_A_T[j][i] > RECURSOS_A_T[max][i] + CUSTO_A_T[max][i])
				max = j;
		}
		
		// Calcula um fator de acordo com o valor alfa
		fator = RECURSOS_A_T[min][i] + alfa * (RECURSOS_A_T[max][i] - RECURSOS_A_T[min][i]);

		// procura o agente que tem maior proximidade com o fator
		min = 0;
		for (j = 1; j < QUANT_AGENTES; j++) {
			if (abs(RECURSOS_A_T[j][i] - fator) < abs(RECURSOS_A_T[min][i] - fator))
				min = j;
		}
		
		// Atribui a esta tarefa
		(*s)->tarefas[i] = min;

		// Calcula a quantidade de recusto utilizado ao atribuir a nova tarefa ao agente.
		(*s)->excesso[min] += RECURSOS_A_T[min][i];

		// Calcula o custo daquela tarefa
		(*s)->custo += CUSTO_A_T[min][i];
	}

	// Verifica se a solução gerada é válida
	for (j = 0; j < QUANT_AGENTES; j++) {
		sum_recursos += (*s)->excesso[j];

		if ((*s)->excesso[j] > CAPAC_AGENTES[j]) {
			solucao_invalida = 1;
		}
	}

	// Calcula o fator avaliação 
	if (solucao_invalida) 
		(*s)->avaliacao = ((double) sum_recursos ) * 1000000;
	else {
		(*s)->avaliacao = ((double) (*s)->custo);
	}
}
			\end{minted}

		Como já descrita uma vez na Seção \ref{sec:otimizacao}, o procedimento \textit{Construção Randomicamente Gulosa} também ganhou o processamento de avaliação otimizada sem a necessidade de solicitar o procedimento \verb|Avalia_Solucao(Solucao)|.
	
	
	\subsection{Arquivos de Configuração de Execução}
		O arquivo de configuração deste algoritmo é composto dos seguintes itens:
		
		\begin{enumerate}
			\item \textbf{Valor Inteiro de Tempo de Processamento:} Quantidade de segundos de processamento; e
			
			\item \textbf{Quantidade de Iterações:} Número de iterações de busca realizados na solução gerada.
		\end{enumerate}
		
		A configuração utilizada no problema foi:
		
		\begin{verbatim}
		60
		5000000
		\end{verbatim}



%%%%%%%%%%%%%%%%%%%%%%%%%%%%%%%%%%%%%%%%%%%%%%%%%%%%%%%%%%%%%%%%%%%%%%%%%%%%%%%%%%%%%%%%%%%%%%%%
%                                                                                              %
%                                      Linha de Comando                                        %
%                                                                                              %
%%%%%%%%%%%%%%%%%%%%%%%%%%%%%%%%%%%%%%%%%%%%%%%%%%%%%%%%%%%%%%%%%%%%%%%%%%%%%%%%%%%%%%%%%%%%%%%%

\section{Entrada de Argumentos por Linha de Comando}

	Como argumentos de entrada para a execução do programa, definiu-se os seguintes parâmetros:
	
	\begin{enumerate}
		\item Nome do Programa;
		\item Arquivo de Configuração do Algoritmo;
		\item Aquivo com a instância;
		\item \textit{Seed} para o gerador aleatório e reprodução de experimentos.
	\end{enumerate}

	Estes podem ser acessados pelo comando:
	
	\verb|./nome_do_programa arq_conf arq_instancia_OR seed|.



%%%%%%%%%%%%%%%%%%%%%%%%%%%%%%%%%%%%%%%%%%%%%%%%%%%%%%%%%%%%%%%%%%%%%%%%%%%%%%%%%%%%%%%%%%%%%%%%
%                                                                                              %
%                                       Experimentação                                         %
%                                                                                              %
%%%%%%%%%%%%%%%%%%%%%%%%%%%%%%%%%%%%%%%%%%%%%%%%%%%%%%%%%%%%%%%%%%%%%%%%%%%%%%%%%%%%%%%%%%%%%%%
	
\section{Experimentação} \label{sec:exec}
	Para a coleta de resultados, cada algoritmo foi executado no mesmo intervalo de tempo e seu resultado foi comparado com os demais incluindo os valores de ótimo.

	Cada algoritmo realizou-se 10 (dez) iterações em cada uma das 18 instâncias selecionadas para testes com temo médio de um minuto de execução. Elas serão descritas a seguir.

	Utilizou-se também da opção \verb|-Ofast| para a compilação de forma otimizada em todos os algoritmos.
	
	
	\subsection{Ambiente de \textit{Hardware} e \textit{Software} Utilizado para Compilação}
		A descrição do ambiente de testes é descrito na Tabela \ref{tab:arq}.
		
		\begin{table}[H]
			\caption{Tabela com as informações de ambiente de execução do trabalho realizado.}
			\centering \label{tab:arq}
			\begin{tabular}{l|l}
				\hline
				\textbf{Item}                & \textbf{Descrição} \\ \hline \hline
				Processador         & 1 Processador Intel Core i7 - 2,9 GHz         \\
				Núcleos             & 4 Núcleos \\
				Cache L2 (por Núcleo) & 256 KB \\
				Cache L3            & 4 MB \\
				Memória RAM         & 10 GB DDR3        \\
				Arquitetura         & Arquitetura de von Neumann         \\
				Sistema Operacional & OS X 10.11.4 (15E65)         \\
				Versão do Kernel    & Darwin 15.4.0 \\
				Compilador          & Apple LLVM version 7.3.0 (clang-703.0.31)         \\\hline
			\end{tabular}
		\end{table}
	
	\subsection{Análise de Código}
		Como o computador utilizado para exeuções possui como compilador nativo o \textit{clang-703.0.31} da LLVM, então pôde-se executar os analizadores de código em tempo de compilação já corrigindo os respectivos erros por meio de ferramentas como o comando \verb|--analyze| e a ferramenta \textit{Clang Static Analyser}.
	
	\subsection{Instâncias}
		As instâncias disponibilizadas para testes são descritas na Tabela \ref{tab:inst}.
	
		\begin{table}[H]
			\centering
			\caption{Tabela com as informações das Instâncias.} \label{tab:inst}
			\begin{tabular}{cc|cc}
				\hline
				\textbf{Tipo} & \textbf{Nome do Arquivo} & \textbf{Quant. de Agentes} & \textbf{Quant. de Tarefas} \\ \hline \hline
				A & \textit{a05100} & 5   & 100 \\
				A & \textit{a10100} & 10  & 100 \\
				A & \textit{a20100} & 20  & 100 \\
				A & \textit{a05200} & 5   & 200 \\
				A & \textit{a10200} & 10  & 200 \\
				A & \textit{a20200} & 20  & 200 \\ \hline
				C & \textit{c05100} & 5   & 100 \\
				C & \textit{c10100} & 10  & 100 \\
				C & \textit{c20100} & 20  & 100 \\
				C & \textit{c05200} & 5   & 200 \\
				C & \textit{c10200} & 10  & 200 \\
				C & \textit{c20200} & 20  & 200 \\ \hline
				E & \textit{e05100} & 5   & 100 \\
				E & \textit{e10100} & 10  & 100 \\
				E & \textit{e20100} & 20  & 100 \\
				E & \textit{e05200} & 5   & 200 \\
				E & \textit{e10200} & 10  & 200 \\
				E & \textit{e20200} & 20  & 200 \\ \hline
			\end{tabular}
		\end{table}
		


%%%%%%%%%%%%%%%%%%%%%%%%%%%%%%%%%%%%%%%%%%%%%%%%%%%%%%%%%%%%%%%%%%%%%%%%%%%%%%%%%%%%%%%%%%%%%%%%
%                                                                                              %
%                                           Resultados                                         %
%                                                                                              %
%%%%%%%%%%%%%%%%%%%%%%%%%%%%%%%%%%%%%%%%%%%%%%%%%%%%%%%%%%%%%%%%%%%%%%%%%%%%%%%%%%%%%%%%%%%%%%%

\section{Resultados}
	Cada algoritmo executará dez vezes cada instância no intervalo de tempo de um minuto alterando os valores de \textit{seed}. Após a execução, obteve os seguintes resultados exibidos na Tabela \ref{tab:minavgmax} de cada algoritmo em cada instância.

	{ \scriptsize
		\begin{longtable}{cc|cccc|cc}
			\caption{Tabela com resultados estatísticos obtidos.} \label{tab:minavgmax} \\
			\hline
			\textbf{Algor.} & \textbf{Arquivo} & \textbf{Min} & \textbf{Média} & \textbf{Max} & \textbf{Desvio Padrão} & \textbf{Ótimo Literatura} & \textbf{\% Acerto} \\ \hline \hline
			GA                 & \textit{a05100}  & 1698      & 1698           & 1698         & 0                      & 1698                            & 100 \% \\
			GA                 & \textit{a10100}  & 1360      & 1360           & 1360         & 0                      & 1360                            & 100 \% \\
			GA                 & \textit{a20100}  & 1158      & 1158           & 1158         & 0                      & 1158                            & 100 \% \\
			GA                 & \textit{a05200}  & 3235      & 3235           & 3235         & 0                      & 3235                            & 100 \% \\
			GA                 & \textit{a10200}  & 2623      & 2623           & 2623         & 0                      & 2623                            & 100 \% \\
			GA                 & \textit{a20200}  & 2339      & 2339           & 2339         & 0                      & 2339                            & 100 \% \\
			GA                 & \textit{c05100}  & 1982      & 1982           & 1982         & 0                      &  1931                           & 97.42684 \% \\
			GA                 & \textit{c10100}  & 1439      & 1439           & 1439         & 0                      &  1402                           & 97.42877 \% \\
			GA                 & \textit{c20100}  & 1281      & 1281           & 1282         & 0.421637               &  1243                           & 97.03357 \% \\
			GA                 & \textit{c05200}  & 3552      & 3553           & 3554         & 0.9660918              &  3456                           & 97.2973 \% \\
			GA                 & \textit{c10200}  & 3006      & 3006           & 3006         & 0                      &  2806                           & 93.34664 \% \\
			GA                 & \textit{c20200}  & 2586      & 2587           & 2590         & 1.414214               &  2391                           & 92.4594 \% \\
			GA                 & \textit{e05100}  & 13900     & 13900          & 13900        & 0                      &   12673                         & 91.20547 \% \\
			GA                 & \textit{e10100}  & 13920     & 13920          & 13920        & 0                      &   11568                         & 83.11539 \% \\
			GA                 & \textit{e20100}  & 10170     & 10170          & 10170        & 0                      &   8431                          & 82.91699 \% \\
			GA                 & \textit{e05200}  & 26820     & 26820          & 26820        & 0                      &   24927                         & 92.9453 \% \\
			GA                 & \textit{e10200}  & 27190     & 27190          & 27190        & 0                      &   23302                         & 85.70693 \% \\
			GA                 & \textit{e20200}  & 27580     & 27580          & 27580        & 0                      &   22377                         & 81.12606 \% \\ \hline
			SA                 & \textit{a05100}  & 1698      & 1698           & 1698         & 0                      & 1698                            & 100 \% \\
			SA                 & \textit{a10100}  & 1360      & 1360           & 1360         & 0                      & 1360                            & 100 \% \\
			SA                 & \textit{a20100}  & 1158      & 1158           & 1158         & 0                      & 1158                            & 100 \% \\
			SA                 & \textit{a05200}  & 3235      & 3235           & 3235         & 0                      & 3235                            & 100 \% \\
			SA                 & \textit{a10200}  & 2623      & 2623           & 2623         & 0                      & 2623                            & 100 \% \\
			SA                 & \textit{a20200}  & 2339      & 2339           & 2339         & 0                      & 2339                            & 100 \% \\
			SA                 & \textit{c05100}  & 1937      & 1937           & 1937         & 0                      &  1931                           & 99.69024 \% \\
			SA                 & \textit{c10100}  & 1415      & 1415           & 1415         & 0                      &  1402                           & 99.08127 \% \\
			SA                 & \textit{c20100}  & 1264      & 1264           & 1264         & 0                      &  1243                           & 98.33861 \% \\
			SA                 & \textit{c05200}  & 3460      & 3460           & 3460         & 0                      &  3456                           & 99.88439 \% \\
			SA                 & \textit{c10200}  & 2838      & 2838           & 2838         & 0                      &  2806                           & 98.87245 \% \\
			SA                 & \textit{c20200}  & 2413      & 2413           & 2413         & 0                      &  2391                           & 99.08827 \% \\
			SA                 & \textit{e05100}  & 12760     & 12760          & 12760        & 0                      &   12673                         & 99.30262 \% \\
			SA                 & \textit{e10100}  & 11830     & 11830          & 11830        & 0                      &   11568                         & 97.7605 \% \\
			SA                 & \textit{e20100}  & 8837      & 8837           & 8837         & 0                      &   8431                          & 95.40568 \% \\
			SA                 & \textit{e05200}  & 25110     & 25110          & 25110        & 0                      &   24927                         & 99.26725 \% \\
			SA                 & \textit{e10200}  & 23820     & 23820          & 23820        & 0                      &   23302                         & 97.80483 \% \\
			SA                 & \textit{e20200}  & 23570     & 23570          & 23570        & 0                      &   22377                         & 94.93445 \% \\ \hline
			Reinício           & \textit{a05100}  & 1698      & 1698           & 1698         & 0                      & 1698                            & 100 \% \\
			Reinício           & \textit{a10100}  & 1360      & 1360           & 1360         & 0                      & 1360                            & 100 \% \\
			Reinício           & \textit{a20100}  & 1158      & 1158           & 1158         & 0                      & 1158                            & 100 \% \\
			Reinício           & \textit{a05200}  & 3235      & 3235           & 3235         & 0                      & 3235                            & 100 \% \\
			Reinício           & \textit{a10200}  & 2623      & 2623           & 2623         & 0                      & 2623                            & 100 \% \\
			Reinício           & \textit{a20200}  & 2339      & 2339           & 2339         & 0                      & 2339                            & 100 \% \\
			Reinício           & \textit{c05100}  & 1953      & 1953           & 1953         & 0                      &  1931                           & 98.87353  \% \\
			Reinício           & \textit{c10100}  & 1433      & 1433           & 1433         & 0                      &  1402                           & 97.83671  \% \\
			Reinício           & \textit{c20100}  & 1264      & 1264           & 1264         & 0                      &  1243                           & 98.33861  \% \\
			Reinício           & \textit{c05200}  & 3503      & 3503           & 3503         & 0                      &  3456                           & 98.65829 \% \\
			Reinício           & \textit{c10200}  & 2852      & 2852           & 2852         & 0                      &  2806                           & 98.3871  \% \\
			Reinício           & \textit{c20200}  & 2445      & 2445           & 2445         & 0                      &  2391                           & 97.79141  \% \\
			Reinício           & \textit{e05100}  & 13950     & 13950          & 13950        & 0                      &   12673                         & 90.8589  \% \\
			Reinício           & \textit{e10100}  & 13200     & 13200          & 13200        & 0                      &   11568                         & 87.643  \% \\
			Reinício           & \textit{e20100}  & 9534      & 9534           & 9534         & 0                      &   8431                          & 88.43088  \% \\
			Reinício           & \textit{e05200}  & 28690     & 28690          & 28690        & 0                      &   24927                         & 86.87485  \% \\
			Reinício           & \textit{e10200}  & 27230     & 27230          & 27230        & 0                      &   23302                         & 85.58416  \% \\
			Reinício           & \textit{e20200}  & 25990     & 25990          & 25990        & 0                      &   22377                         & 86.09187 \% \\ \hline
			GRASP              & \textit{a05100}  & 1698      & 1698           & 1698         & 0                      & 1698                            & 100                  \% \\
			GRASP              & \textit{a10100}  & 1360      & 1360           & 1360         & 0                      & 1360                            & 100                  \% \\
			GRASP              & \textit{a20100}  & 1158      & 1158           & 1158         & 0                      & 1158                            & 100                  \% \\
			GRASP              & \textit{a05200}  & 3235      & 3235           & 3235         & 0                      & 3235                            & 100                  \% \\
			GRASP              & \textit{a10200}  & 2623      & 2623           & 2623         & 0                      & 2623                            & 100                  \% \\
			GRASP              & \textit{a20200}  & 2339      & 2339           & 2339         & 0                      & 2339                            & 100                  \% \\
			GRASP              & \textit{c05100}  & 1955      & 1955           & 1955         & 0                      &  1931                           & 98.77238                       \% \\
			GRASP              & \textit{c10100}  & 1419      & 1419           & 1419         & 0                      &  1402                           & 98.80197                       \% \\
			GRASP              & \textit{c20100}  & 1268      & 1268           & 1268         & 0                      &  1243                           & 98.02839                       \% \\
			GRASP              & \textit{c05200}  & 3490      & 3490           & 3490         & 0                      &  3456                           & 99.02579                       \% \\
			GRASP              & \textit{c10200}  & 2860      & 2860           & 2860         & 0                      &  2806                           & 98.11189                       \% \\
			GRASP              & \textit{c20200}  & 2457      & 2457           & 2457         & 0                      &  2391                           & 97.3138                     \% \\
			GRASP              & \textit{e05100}  & 13840     & 13840          & 13840        & 0                      &   12673                         & 91.58777                       \% \\
			GRASP              & \textit{e10100}  & 12760     & 12760          & 12760        & 0                      &   11568                         & 90.6228                       \% \\
			GRASP              & \textit{e20100}  & 9164      & 9164           & 9164         & 0                      &   8431                          & 92.00131                       \% \\
			GRASP              & \textit{e05200}  & 29050     & 29050          & 29050        & 0                      &   24927                         & 85.81314                       \% \\
			GRASP              & \textit{e10200}  & 26660     & 26660          & 26660        & 0                      &   23302                         & 87.39124                       \% \\
			GRASP              & \textit{e20200}  & 25580     & 25580          & 25580        & 0                      &   22377                         & 87.47508           \% \\ \hline
		\end{longtable}
	}

	
	\subsection{Resultados em Gráficos}
		Abaixo serão exibidos as comparações de cada algoritmo em cada instância. Para melhor visualização, foi adicionado uma linha horizontal azulada representando o ótimo da literatura.
	
		\subsubsection{Instância \textit{a05100}}
			\begin{figure}[H]
				\centering
				\includegraphics[width=1\linewidth]{img/1.png}
				\caption{BoxPlot da Instância \textit{a05100}.}
				\label{fig:a05100}
			\end{figure}
	
		\subsubsection{Instância \textit{a05200}}
			\begin{figure}[H]
				\centering
				\includegraphics[width=1\linewidth]{img/2.png}
				\caption{BoxPlot da Instância \textit{a05200}.}
				\label{fig:a05200}
			\end{figure}
	
		\subsubsection{Instância \textit{a10100}}
			\begin{figure}[H]
				\centering
				\includegraphics[width=1\linewidth]{img/3.png}
				\caption{BoxPlot da Instância \textit{a10100}.}
				\label{fig:a10100}
			\end{figure}
	
		\subsubsection{Instância \textit{a10200}}
			\begin{figure}[H]
				\centering
				\includegraphics[width=1\linewidth]{img/4.png}
				\caption{BoxPlot da Instância \textit{a10200}.}
				\label{fig:a10200}
			\end{figure}
	
		\subsubsection{Instância \textit{a20100}}
			\begin{figure}[H]
				\centering
				\includegraphics[width=1\linewidth]{img/5.png}
				\caption{BoxPlot da Instância \textit{a20100}.}
				\label{fig:a20100}
			\end{figure}
	
		\subsubsection{Instância \textit{a20200}}
			\begin{figure}[H]
				\centering
				\includegraphics[width=1\linewidth]{img/6.png}
				\caption{BoxPlot da Instância \textit{a20200}.}
				\label{fig:a20200}
			\end{figure}
	

	
		\subsubsection{Instância \textit{c05100}}
			\begin{figure}[H]
				\centering
				\includegraphics[width=1\linewidth]{img/7.png}
				\caption{BoxPlot da Instância \textit{c05100}.}
				\label{fig:c05100}
			\end{figure}
	
		\subsubsection{Instância \textit{c05200}}
			\begin{figure}[H]
				\centering
				\includegraphics[width=1\linewidth]{img/8.png}
				\caption{BoxPlot da Instância \textit{c05200}.}
				\label{fig:c05200}
			\end{figure}
	
		\subsubsection{Instância \textit{c10100}}
			\begin{figure}[H]
				\centering
				\includegraphics[width=1\linewidth]{img/9.png}
				\caption{BoxPlot da Instância \textit{c10100}.}
				\label{fig:c10100}
			\end{figure}
	
		\subsubsection{Instância \textit{c10200}}
			\begin{figure}[H]
				\centering
				\includegraphics[width=1\linewidth]{img/10.png}
				\caption{BoxPlot da Instância \textit{c10200}.}
				\label{fig:c10200}
			\end{figure}
	
		\subsubsection{Instância \textit{c20100}}
			\begin{figure}[H]
				\centering
				\includegraphics[width=1\linewidth]{img/11.png}
				\caption{BoxPlot da Instância \textit{c20100}.}
				\label{fig:c20100}
			\end{figure}
	
		\subsubsection{Instância \textit{c20200}}
			\begin{figure}[H]
				\centering
				\includegraphics[width=1\linewidth]{img/12.png}
				\caption{BoxPlot da Instância \textit{c20200}.}
				\label{fig:c20200}
			\end{figure}


	
		\subsubsection{Instância \textit{e05100}}
			\begin{figure}[H]
				\centering
				\includegraphics[width=1\linewidth]{img/13.png}
				\caption{BoxPlot da Instância \textit{e05100}.}
				\label{fig:e05100}
			\end{figure}
	
		\subsubsection{Instância \textit{e05200}}
			\begin{figure}[H]
				\centering
				\includegraphics[width=1\linewidth]{img/14.png}
				\caption{BoxPlot da Instância \textit{e05200}.}
				\label{fig:e05200}
			\end{figure}
	
		\subsubsection{Instância \textit{e10100}}
			\begin{figure}[H]
				\centering
				\includegraphics[width=1\linewidth]{img/15.png}
				\caption{BoxPlot da Instância \textit{e10100}.}
				\label{fig:e10100}
			\end{figure}
	
		\subsubsection{Instância \textit{e10200}}
			\begin{figure}[H]
				\centering
				\includegraphics[width=1\linewidth]{img/16.png}
				\caption{BoxPlot da Instância \textit{e10200}.}
				\label{fig:e10200}
			\end{figure}
	
		\subsubsection{Instância \textit{e20100}}
			\begin{figure}[H]
				\centering
				\includegraphics[width=1\linewidth]{img/17.png}
				\caption{BoxPlot da Instância \textit{e20100}.}
				\label{fig:e20100}
			\end{figure}
	
		\subsubsection{Instância \textit{e20200}}
			\begin{figure}[H]
				\centering
				\includegraphics[width=1\linewidth]{img/18.png}
				\caption{BoxPlot da Instância \textit{e20200}.}
				\label{fig:e20200}
			\end{figure}
			
	
	\subsection{Estudo Estatísticos}
		Como análise comparativa final, foi confeccionado uma tabela comparando a média das distância de todos os algoritmos em todas as instâncias executadas. O resultado é exibido na Tabela \ref{tab:comparacaoFinal}.
		
	\begin{table}[h]
		\caption{Análise final de proximidade de cada algoritmo em relação ao ótimo.} \label{tab:comparacaoFinal}
		\centering
	    \begin{tabular}{c|c}
	    \hline
	    \textbf{Algoritmos}  &  \textbf{Média Geral em Relação ao Ótimo} \\ \hline \hline
	    Algoritmo Genético   &  94.00048 \%                \\
	    Recozimento Simulado &  98.85725 \%                \\
	    Método Reinício      &  95.29829 \%                \\
	    GRASP                &  95.83031 \%                \\ \hline
	    \end{tabular}
	\end{table}
	
	É possível concluir que todos os algoritmos teve um resultado médio acima de 94\% em todas as instâncias, sendo uma boa porcentagem quando necessita-se de um valor próximo ao ótimo em problemas combinatórios.
	
	Um item que deve-se atentar é no algoritmo de Recozimento Simulado que obteve uma média geral de quase 99\% de proximidade ao ótimo global encontrado pela literatura.
	

%%%%%%%%%%%%%%%%%%%%%%%%%%%%%%%%%%%%%%%%%%%%%%%%%%%%%%%%%%%%%%%%%%%%%%%%%%%%%%%%%%%%%%%%%%%%%%%%
%                                                                                              %
%                                           Conclusão                                         %
%                                                                                              %
%%%%%%%%%%%%%%%%%%%%%%%%%%%%%%%%%%%%%%%%%%%%%%%%%%%%%%%%%%%%%%%%%%%%%%%%%%%%%%%%%%%%%%%%%%%%%%%

\section{Comentários Finais}\label{sec:figs}
	A primeira conclusão a ser feita é a diferença do Algoritmo Genético em relação aos outros. Diferentemente do Genético, os três algoritmos compartilham da mesma natureza de procura que é baseada na geração de indivíduos aleatórios e busca local em seus vizinhos o que implica exatamente nos dados amostrados. Os três outros algoritmos (Recozimento Simulado, Método de Reinício e GRASP) utilizam de um processamento simples. Com não é necessário realizar muitas operações para completar uma iteração de seu processamento, eles permitem que sejam realizados mais cálculos por intervalo de tempo, proporcionando uma vantagem maior quando o processador de testes consegue realizar muito processamento por segundo. Estes algoritmos, nesta vantagem, permitem que vasculhem uma gama maior de locais a procura de um ótimo/solução boa favorável, o que realmente acontece. Ao contrário, o Algoritmo Genético é um algoritmo lento e com curva de otimização lenta já que é preciso realizar várias operações em seu processamento.
	
	Um item importante a ser abordado é o método de avaliação presente nos métodos \verb|Avalia_Solucao()| e \verb|Gera_Vizinho()|. Como o método possui duas funções diferentes (uma pra soluções válidas e outra para inválidas), é possível perceber que resultados válidos/factíveis são gerados de forma mais rápida tornando o algoritmo mais eficaz. O método \verb|Avalia_Solucao()| é invocado somente na construção de uma solução incial aleatória nos algoritmos Recozimento Simulado, Método de Reinício e GRASP melhorando ainda mais sua performance ao utilizar o procedimento de geração de vizinho otimizada (no qual não necessita de invocar o método de avaliação).
	
	Os parâmetros de todos os algoritmos foram mantidos os mesmos para todas as instâncias para generalizar seu desempenho e eficiência na procura de boas soluções sem depender de determinadas características específicas de determinada instância.

	Também foi possível concluir que a literatura utiliza um tempo maior de execução para encontrar soluções ótimas, o que difere deste trabalho. Neste, foi determinado um tempo de um minuto por execução o que restringe ainda mais a proximidade do ótimo global. Entretanto, todos os algoritmos obtiveram excelente resultados mesmo em tempo de execução curto.
	
	
\section{Código dos Algoritmos}
	Os algoritmos \textit{shell}, Algoritmo Genético, Recozimento Simulado, Método Reinício e GRASP, utilizados no trabalho, estão situados nas páginas \pageref{cod:shell}, \pageref{cod:ga}, \pageref{cod:sa}, \pageref{cod:greedy}, \pageref{cod:grasp}, respectivamente.
	
	\subsection{\textit{Shell Script}} \label{cod:shell}

\begin{minted} [ frame=lines, framesep=2mm, tabsize=3, breaklines=true, baselinestretch=1.2, linenos, fontsize=\footnotesize ]{shell}
#!/bin/bash

eval "gcc -Ofast gap.c GAP-Genetic.c -o GAP-Genetic.o"
eval "gcc -Ofast gap.c GAP-GRASP.c   -o GAP-GRASP.o"
eval "gcc -Ofast gap.c GAP-Greedy.c  -o GAP-Greedy.o"
eval "gcc -Ofast gap.c GAP-SA.c      -o GAP-SA.o"

instancias=( 
  ../Instancias/gap_a/a05100
  ../Instancias/gap_a/a05200
  ../Instancias/gap_a/a10100
  ../Instancias/gap_a/a10200
  ../Instancias/gap_a/a20100
  ../Instancias/gap_a/a20200

  ../Instancias/gap_c/c05100
  ../Instancias/gap_c/c05200
  ../Instancias/gap_c/c10100
  ../Instancias/gap_c/c10200
  ../Instancias/gap_c/c20100
  ../Instancias/gap_c/c20200

  ../Instancias/gap_e/e05100
  ../Instancias/gap_e/e05200
  ../Instancias/gap_e/e10100
  ../Instancias/gap_e/e10200
  ../Instancias/gap_e/e20100
  ../Instancias/gap_e/e20200
  )

algoritmos=( GAP-Genetic.o GAP-GRASP.o GAP-Greedy.o GAP-SA.o )

configuracoes=( p_ga.txt p_grasp.txt p_greedy.txt p_sa.txt )

echo "Quantas iteracoes?"
read quantidade_iteracoes;
echo


for indice_algoritmo in "${!algoritmos[@]}"
do
	echo $indice_algoritmo

  for instancia in "${instancias[@]}"
  do
  	echo $instancia

    for (( i = 0; i < "$quantidade_iteracoes"; i++ )); do
      echo "$i"

      cmd="./${algoritmos[$indice_algoritmo]} ${configuracoes[$indice_algoritmo]} $instancia $i"
      date
      echo $cmd
      $cmd
      echo
      echo "-------------------------------------------------------------"
      echo
    done
    echo

  done
  echo

done


\end{minted}

	\subsection{Códigos em \textit{R}}
		\subsubsection{Procedimento Estatístico}
		
		\begin{minted} [ frame=lines, framesep=2mm, tabsize=3, breaklines=true, baselinestretch=1.2, linenos, fontsize=\footnotesize ]{r}

GeraDados <- function(diretorio) {
  arq_gen <- read.table(diretorio)
  vet_gen <- rep(arq_gen$V1);
  
  desloca <- 1;
  cat("\tMin. 1st Qu.  Median    Mean 3rd Qu.    Max.\n")
    
  
  
  otimo <- c(1698   ,
             1360   ,
             1158   ,
             3235   ,
             2623   ,
             2339   ,
             1931  ,
             1402  ,
             1243  ,
             3456  ,
             2806  ,
             2391  ,
             12673,
             11568,
             8431 ,
             24927,
             23302,
             22377)
  
  # 3 * 6 arquivos (a, c, d)
  for (j in 1:18) {
    cat(j, "\t");
    vetor_buffer <- vet_gen[desloca:(desloca + 9)];
    cat(summary(vetor_buffer), "\t\t")
    
    cat(sd(vetor_buffer))
    cat("\n")
    
    desloca = desloca + 10;
  }
  desloca = 1;
  for (j in 1:18) {
    #cat(j, "\t");
    vetor_buffer <- vet_gen[desloca:(desloca + 9)];
    cat(100 * (otimo[j] / min(vetor_buffer)))
    cat("\n")
    
    desloca = desloca + 10;
  }
  
}

GeraDados("out_genetico.txt")
GeraDados("out_simulated_annealing.txt")
GeraDados("out_reinicio.txt")
GeraDados("out_grasp.txt")

		\end{minted}
		
		\subsubsection{Gráficos}
		
		\begin{minted} [ frame=lines, framesep=2mm, tabsize=3, breaklines=true, baselinestretch=1.2, linenos, fontsize=\footnotesize ]{r}
otimo <- c(1698   ,
           1360   ,
           1158   ,
           3235   ,
           2623   ,
           2339   ,
           1931  ,
           1402  ,
           1243  ,
           3456  ,
           2806  ,
           2391  ,
           12673,
           11568,
           8431 ,
           24927,
           23302,
           22377)
GeraDados <- function(d_ga, d_sa, d_r, d_gr) {
  arq_ga <- read.table(d_ga)
  arq_sa <- read.table(d_sa)
  arq_r <- read.table(d_r )
  arq_gr <- read.table(d_gr)
  
  ga <- rep(arq_ga$V1);
  sa <- rep(arq_sa$V1);
  r  <- rep(arq_r$V1);
  gr <- rep(arq_gr$V1);
  
  desloca <- 1;
  
  
  for (j in 1:18) {
    #png(paste(d_ga, "_", j, ".png", sep=""), width = 840, height = 480, units = "px");
    png(paste(j, ".png", sep=""), width = 840, height = 480, units = "px");
    
    boxplot(ga[desloca:(desloca + 9)], 
            sa[desloca:(desloca + 9)], 
            r[desloca:(desloca + 9)], 
            gr[desloca:(desloca + 9)],
            ylim = c(otimo[j] - 1, 
                     max(max(ga[desloca:(desloca + 9)]), max(sa[desloca:(desloca + 9)]), max(r[desloca:(desloca + 9)]), max(gr[desloca:(desloca + 9)])) + 1),
            main = paste("Instância ", j, sep = ""),
            xlab = "Algoritmos",
            ylab = "Custo", 
            col = "orange",
            border = "brown",
            names = c("Genético", "Simulated Annealing", "Reinício", "GRASP"));
    abline(h = otimo[j], col = "blue");
    
    dev.off()
    
    desloca = desloca + 10;
  }
}

GeraDados("out_genetico.txt", "out_simulated_annealing.txt", "out_reinicio.txt", "out_grasp.txt")

		\end{minted}

	\subsection{Códigos em \textit{C}}

	\subsubsection{\textit{gap.h}} \label{cod:mod}

		\begin{minted} [ frame=lines, framesep=2mm, tabsize=3, breaklines=true, baselinestretch=1.2, linenos, fontsize=\footnotesize ]{c}

/*
 * Trabalho de Projeto e Análise de Algoritmo
 * Período 16.1
 * 
 * Desenvolver Metaheurísticas para o Problema de Alocação Generalizada
 * 
 * Funções Genéricas do Problam GAP.
 * Data: 01/08/2016.
 * Distribuição Livre, desde que referenciando o autor.
 * 
 * Professor: Haroldo Santos
 * 
 * Autor do Trabalho: Rodolfo Labiapari Mansur Guimarães
 */

#include <time.h>

/*
 * Estrutura de dados para armazenamento das informações de conjunto de 
 *	soluções.
 * 
 * excesso - valor com a capacidade atual de cada agente desta solução
 * tarefas - Vetor com tamanho Tarefas onde cada endereço é indicado o agente
 *	responsável por tal
 * avaliação - Valor fitness da solução (sum_excesso + custo) * penalidade
 * custo - custo total desta solução
 */
typedef struct Struct_Solucao {
	int * excesso;
	int * tarefas;
	double avaliacao;
	double custo;
} Solucao;

int QUANT_TAREFAS   ; // Quantidade de tarefas
int QUANT_AGENTES   ; // Quantidade de agentes
int * CAPAC_AGENTES ; // Vetor com capacidade máxima de cada gente.
int ** RECURSOS_A_T ; // Matriz de recursos[a,t]
int ** CUSTO_A_T    ; // Matriz de custo [a,t]


int    TAM_POP    ; // Tamanho da população do algoritmo
int    ITERACOES  ; // Quantidade de iterações
float  TAX_CRUZAM ; // Porcentagem de cruzamentos a serem realizados
float  TAX_MUT    ; // Porcentagem dos dados do filho que serão mutados

char IMPRIMIR ; // Variável que permite impressão na tela.
int SECONDS   ; // Tempo lido pelo arquivo de configuração
int MAXIteracoes ; // Quantidade de iterações daquela temperatura
FILE * out    ; // Arquivo para gravação de dados permanente
time_t endwait, start; // Variáveis de tempo para cálculo do intervalo de 
	//tempo de execução

double Avalia_Solucao(Solucao * sol) ;
void Gera_Vizinho(Solucao * atual, Solucao ** proxima) ;
Solucao * Instancia_Solucao() ;
void Gera_Solucao_Aleatoria(Solucao ** s) ;
char Teste_Aceita_Nova_Solucao(double temp, Solucao ** atual, Solucao * proxima) ;
void Aceita_Nova_Solucao(Solucao ** atual, Solucao * proxima) ;
void Le_Instancia(char * path) ;
void Imprime_Solucao(Solucao * ind) ;
void Imprime_Status(double t, Solucao * m) ;
void Desaloca_Solucao(Solucao ** p) ;
Solucao * Instancia_Solucao_Aleatoria() ;
void Desaloca_Populacao(Solucao *** p) ;
void Imprime_Instancia() ;
		\end{minted}


	\subsubsection{\textit{gap.c}}
		\begin{minted} [ frame=lines, framesep=2mm, tabsize=3, breaklines=true, baselinestretch=1.2, linenos, fontsize=\footnotesize ]{c}

/*
 * Trabalho de Projeto e Análise de Algoritmo
 * Período 16.1
 * 
 * Desenvolver Metaheurísticas para o Problema de Alocação Generalizada
 * 
 * Funções Genéricas do Problam GAP.
 * Data: 01/08/2016.
 * Distribuição Livre, desde que referenciando o autor.
 * 
 * Professor: Haroldo Santos
 * 
 * Autor do Trabalho: Rodolfo Labiapari Mansur Guimarães
 */

#include <stdio.h>
#include <stdlib.h>
#include <math.h>
#include <limits.h>
#include "gap.h"


/*
 * Procedimento que avalia a solução atual.
 * 
 * Se a solução é factível, retorna seu custo real.
 * Se infactível retorna o custo encontrado * 1000000
 */
double Avalia_Solucao(Solucao * sol) {
	int i = 0, capacidade_agentes[QUANT_AGENTES], sum_recursos = 0;
	double custo = 0; char solucao_invalida = 0;
	
	// Define a capacidade inicial utilizada de cada agente com 0
	for (i = 0; i < QUANT_AGENTES; i++)
		capacidade_agentes[i] = 0;

	// Realiza os cálculos de custo e capacidade
	for (i = 0; i < QUANT_TAREFAS; i++) {
		custo += CUSTO_A_T[sol->tarefas[i]][i];
		capacidade_agentes[sol->tarefas[i]] += RECURSOS_A_T[sol->tarefas[i]][i];
	}

	// Verifica se algum agente passou sua capacidade máxima
	for (i = 0; i < QUANT_AGENTES; i++) {
		sol->excesso[i] = capacidade_agentes[i];
		sum_recursos += capacidade_agentes[i];
	
		// Se sim define esta solução como inválida
		if (capacidade_agentes[i] > CAPAC_AGENTES[i])
			solucao_invalida = 1;
	}
	
	sol->custo = custo;
	
	// Caso a solução foi excedida, altera a avaliação do indivíduo tornando-o
		// pior.
	if (solucao_invalida) 
		sol->avaliacao = ((double) sum_recursos ) * 1000000;
	else {
		sol->avaliacao = ((double) custo);
	}
	
	return sol->avaliacao;
}



/*
 * Procedimento de geração de indivíduos por meio do procedimento shift. 
 * 
 * Um detalhe a se atentar é que é realizado shift em somente 1 tarefa da 
 *	solução.
 */
void Gera_Vizinho(Solucao * atual, Solucao ** proxima) {
	int i = 0, j = 0, agente_atual = 0, agente_novo = 0, 
	tarefa_escolhida1 = 0, sum_recursos = 0, 
	quant_alteracoes = 0;
	char solucao_invalida = 0;
	
	// Copia os valores do atual
	for (i = 0; i < QUANT_TAREFAS; i++) {
		(*proxima)->tarefas[i] = atual->tarefas[i];
		if (i < QUANT_AGENTES)
			(*proxima)->excesso[i] = atual->excesso[i];
	}

	(*proxima)->custo = atual->custo;

	
	// Define uma quantidade de alterações pra gerar o vizinho
	quant_alteracoes = 2;
	
	// Altera o indivíduo
	for (i = 0; i < quant_alteracoes; i++) {

		// Escolhe a tarefa que será alterada
		tarefa_escolhida1 = random() % QUANT_TAREFAS;
		agente_atual = (*proxima)->tarefas[tarefa_escolhida1]; 

		// Gera um novo agente pra ela e certifica que ele é diferente do anterior.
		do {
			agente_novo = random() % QUANT_AGENTES;
		} while (agente_novo == agente_atual);

		// Atribui o novo agente à tarefa
		(*proxima)->tarefas[tarefa_escolhida1] = agente_novo;

		// Procedimento Otimizado:
			// A cada alteração, realiza-se a alteração dos valores da nova geração.
			// Como este novo vizinho não é feito do zero e sim sobre cópia de um anterior,
			// basta alterar os valores herdados do seu anterior, atualiando em O(1)

		// Sendo assim, atualiza o excesso de cada agente
			// Retira recurso do agente que ficou livre
		(*proxima)->excesso[agente_atual] -= RECURSOS_A_T[agente_atual][tarefa_escolhida1];
			// Acrescenta recurso do agente que recebeu a tarefa atual
		(*proxima)->excesso[agente_novo ] += RECURSOS_A_T[agente_novo ][tarefa_escolhida1]; 

		// Calcula o custo atual desta solução
		(*proxima)->custo += CUSTO_A_T[agente_novo ][tarefa_escolhida1] - CUSTO_A_T[agente_atual][tarefa_escolhida1];
	}

	// Verifica se a solução gerada é válida
	for (j = 0; j < QUANT_AGENTES; j++) {
		sum_recursos += (*proxima)->excesso[j];

		if ((*proxima)->excesso[j] > CAPAC_AGENTES[j]) {
			solucao_invalida = 1;
		}
	}

	// Calcula o fator avaliação 
	if (solucao_invalida) 
		(*proxima)->avaliacao = ((double) sum_recursos ) * 1000000;
	else {
		(*proxima)->avaliacao = ((double)  (*proxima)->custo);
	}
}



/*
 * Procedimento que instancia uma nova solução com seus valores totalmente 
 *	aleatórios.
 */
Solucao * Instancia_Solucao() {
	Solucao * s;
	
	s = calloc(1, sizeof(Solucao));
	
	s->tarefas = calloc(QUANT_TAREFAS, sizeof(int));
	s->excesso = calloc(QUANT_AGENTES, sizeof(int));
	
	// Defini-o como uma solução inicial ruim
	s->avaliacao = INT_MAX;
	s->avaliacao = INT_MAX;
	
	// Retorna a solução
	return s;
}


/*
 * Procedimento que instancia uma nova solução com seus valores totalmente 
 *	aleatórios.
 */
void Gera_Solucao_Aleatoria(Solucao ** s) {
	int i;
	
	for (i = 0; i < QUANT_TAREFAS; i++) {
		(*s)->tarefas[i] = random() % QUANT_AGENTES;
	}
	
	Avalia_Solucao(*s);
}


/*
 * Procedimento que copia as informações de uma solução para a outra de forma
 *	a aceitar aquela solução.
 */
char Teste_Aceita_Nova_Solucao(double temp, Solucao ** atual, Solucao * proxima) {
	int i;
	
	// Verifica se a solução atual é melhor que a atual.
	if ((*atual)->avaliacao > 2 * 1000000 || 
				((proxima->avaliacao < 2 * 1000000) && 
				(proxima->custo < (*atual)->custo))) {
		
		//Imprime_Status(temp, *atual);

		// Copia a solução
		for (i = 0; i < QUANT_TAREFAS; i++) {
			(*atual)->tarefas[i] = proxima->tarefas[i];
			if (i < QUANT_AGENTES)
				(*atual)->excesso[i] = proxima->excesso[i];
		}

		(*atual)->avaliacao = proxima->avaliacao;
		(*atual)->custo     = proxima->custo;
	
		// Retorna true se tiver aceitado.		
		return 1;
	} else {
		return 0;
	}
}


/*
 * Procedimento que copia as informações de uma solução para a outra de forma
 *	a aceitar aquela solução.
 */
void Aceita_Nova_Solucao(Solucao ** atual, Solucao * proxima) {
	int i = 0;
	
	// Copia a solução 
	for (i = 0; i < QUANT_TAREFAS; i++) {
		(*atual)->tarefas[i] = proxima->tarefas[i];
		if (i < QUANT_AGENTES)
			(*atual)->excesso[i] = proxima->excesso[i];
	}

	(*atual)->avaliacao = proxima->avaliacao;
	(*atual)->custo     = proxima->custo;
}




/*
 * Procedimento que realiza a leitura da instância problema.
 */
void Le_Instancia(char * path) {
	int i, j;
	FILE * f = fopen(path, "r");

	if (f) {
		fscanf(f, "%d", &QUANT_AGENTES);
		fscanf(f, "%d", &QUANT_TAREFAS);

		if (QUANT_AGENTES < 1 || QUANT_TAREFAS < 1) {
			printf("Valores da Instância Negativos!");
			exit(2);
		}

		CAPAC_AGENTES = calloc(QUANT_AGENTES, sizeof (int));

		RECURSOS_A_T = calloc(QUANT_AGENTES, sizeof (int*));
		CUSTO_A_T = calloc(QUANT_AGENTES, sizeof (int*));

		for (i = 0; i < QUANT_AGENTES; i++) {
			RECURSOS_A_T[i] = calloc(QUANT_TAREFAS, sizeof (int));
			CUSTO_A_T[i] = calloc(QUANT_TAREFAS, sizeof (int));
		}

		for (i = 0; i < QUANT_AGENTES; i++) {
			for (j = 0; j < QUANT_TAREFAS; j++) {
				fscanf(f, "%d", &CUSTO_A_T[i][j]);
			}
		}

		for (i = 0; i < QUANT_AGENTES; i++) {
			for (j = 0; j < QUANT_TAREFAS; j++) {
				fscanf(f, "%d", &RECURSOS_A_T[i][j]);
			}
		}

		for (i = 0; i < QUANT_AGENTES; i++)
			fscanf(f, "%d", &CAPAC_AGENTES[i]);
		
		fclose(f);

	} else {
		printf("Erro ao ler Instância!\n");
		exit(-2);
	}

}




/*
 * Procedimento que informa o resultado obtido.
 */
void Imprime_Solucao(Solucao * ind) {
	
	fprintf(out, "\n\t%10lf\t", ind->custo);
	printf("\n\t%10lf\t", ind->custo);	
	
	/*for (i = 0; i < QUANT_TAREFAS; i++) {
		fprintf(out, "%3d ", ind->tarefas[i]);
		printf("%3d ", ind->tarefas[i]);
	}*/
}



/*
 * Procedimento de impressão do status atual da execução.
 */
void Imprime_Status(double t, Solucao * m) {
	int i, sum = 0;
	
	for (i = 0; i < QUANT_AGENTES; i++)
		sum += m->excesso[i];

	//printf("I:%7d\t", it);
	printf("T:%7.4lf\t", t);
	printf("Best(c,a,sum):%9.0lf|%9.0lf|%d\t", m->custo, m->avaliacao, sum);
	
	//printf("AVA(c,a):At:%9.0lf|%9.0lf\tProx:%9.0lf|%9.0lf", a->custo, a->avaliacao, p->custo, p->avaliacao);
	printf("\n");
}



/*
 * Procedimento que instancia uma nova solução com seus valores totalmente 
 *	aleatórios.
 */
Solucao * Instancia_Solucao_Aleatoria() {
	int i;
	Solucao * s;
	
	s = calloc(1, sizeof(Solucao));
	
	s->tarefas = calloc(QUANT_TAREFAS, sizeof(int));
	s->excesso = calloc(QUANT_AGENTES, sizeof(int));
	
	for (i = 0; i < QUANT_TAREFAS; i++) {
		s->tarefas[i] = random() % QUANT_AGENTES;
	}
	
	Avalia_Solucao(s);
	
	return s;
}



/*
 * Procedimento que libera a memória da solução utilizada.
 */
void Desaloca_Solucao(Solucao ** p) {
	
	free((*p)->tarefas);
	free((*p)->excesso);
	
	free(*p);
}



/*
 * Procedimento que realiza liberação de memória completa da população P
 */
void Desaloca_Populacao(Solucao *** p) {
	int i = 0;
	
	// Desaloca uma população inteira
	for (i = 0; i < TAM_POP; i++) {
		if ((*p)[i] != NULL) {
			free((*p)[i]->tarefas);
			free((*p)[i]->excesso);
		}
	}
	
	free(*p);
}



/*
 * Imprime a Instância lida
 */
void Imprime_Instancia() {
	int i = 0, j;

	printf("%d %d\n", QUANT_AGENTES, QUANT_TAREFAS);

	for (i = 0; i < QUANT_AGENTES; i++) {
		for (j = 0; j < QUANT_TAREFAS; j++) 
			printf("%d ", CUSTO_A_T[i][j]);
		printf("\n");
	}
	printf("\n");

	for (i = 0; i < QUANT_AGENTES; i++) {
		for (j = 0; j < QUANT_TAREFAS; j++) 
			printf("%d ", RECURSOS_A_T[i][j]);
		printf("\n");
	}
	
	printf("\n");

	for (i = 0; i < QUANT_AGENTES; i++)
		printf("%d ", CAPAC_AGENTES[i]);

	
	printf("\n");
}
		\end{minted}

	\subsubsection{Algoritmo Genético} \label{cod:ga}

		\begin{minted} [ frame=lines, framesep=2mm, tabsize=3, breaklines=true, baselinestretch=1.2, linenos, fontsize=\footnotesize ]{c}
/*
 * Trabalho de Projeto e Análise de Algoritmo
 * Período 16.1
 * 
 * Desenvolver Metaheurísticas para o Problema de Alocação Generalizada
 * 
 * Algoritmo: Genético.
 * Data: 01/08/2016.
 * Distribuição Livre, desde que referenciando o autor.
 * 
 * Professor: Haroldo Santos
 * 
 * Autor do Trabalho: Rodolfo Labiapari Mansur Guimarães
 */

#include <stdio.h>
#include <stdlib.h>
#include <time.h>
#include "gap.h"



/*
 * Procedimento que realiza a alocação de memória para novas instâncias.
 * 
 * O procedimento instancia a população, mas não os indivíduos.
 */
void Instancia_Populacoes(Solucao *** pop) {
	(*pop) = calloc(TAM_POP, sizeof (Solucao*));
}



/*
 * Procedimento que imprime os dados de toda a população.
 */
void Imprime_Populacao (Solucao ** p) {
	int i = 0, j = 0;
	if (IMPRIMIR) {
		
		printf("\n\n");

		for (i = 0; i < TAM_POP; i++) {
			printf("[%3d] - ", i);
			if (p[i] != NULL) {
				for (j = 0; j < QUANT_TAREFAS; j++) {
					printf("%1d  ", p[i]->tarefas[j]);
				}

				printf("\t%10.1lf\n", p[i]->avaliacao);
			} else 
				printf("\n");

			fflush(stdout);
		}

		fprintf(out, "\n\n");

		for (i = 0; i < TAM_POP; i++) {
			fprintf(out, "[%3d] - ", i);
			if (p[i] != NULL) {
				for (j = 0; j < QUANT_TAREFAS; j++) {
					fprintf(out, "%1d  ", p[i]->tarefas[j]);
				}

				fprintf(out, "\t%10.1lf\n", p[i]->avaliacao);
			} else 
				fprintf(out, "\n");

			fflush(out);
		}
	}
}



/*
 * Procedimento que imprime de forma inteligente, quais índices da população
 *	estão preenchidos e quais estão vazios.
 */
void Imprime_Dados_Populacao (Solucao ** p) {
	int i = 0;
	
	printf("\n\n");
	
	for (i = 0; i < TAM_POP; i++) {
		printf("[%5d] ", i);
		printf("c%5.0lf / sum = a%12.6lf", p[i]->custo, p[i]->avaliacao);
		
		if (p[i]->custo > 5 * 209) {
			printf(" X\n");
		} else {
			printf("\n");
		}
	}
}



/*
 * Procedimento que realiza a leitura dos parâmetros de configuração do 
 *	algoritmo.
 */
void Le_Parametros(char * conf) {
	FILE * f = 0;

	f = fopen(conf, "r");

	if (f) {
		fscanf(f, "%d", &TAM_POP);
		fscanf(f, "%f", &TAX_CRUZAM);
		fscanf(f, "%f", &TAX_MUT);
		fscanf(f, "%d", &SECONDS);
		
		fclose(f);
		
	} else {
		printf("Erro ao ler Configuração!\n");
		exit(-1);
	}
}



/*
 * Procedimento que seleciona indivíduos não repetidos aleatoriamente da 
 *	população anterior adicionando-os na nova e eliminando os indivíduos 
 *	rejeitados.
 */
void Seleciona_Nova_Geracao(Solucao *** atual, Solucao *** proxima) {
	int i = 0, buffer[TAM_POP], indice = 0;
	Solucao ** ind_buffer = 0;
	
	// Inicializa o buffer de usados como 'nenhum item utilizado'
	for (i = 0; i < TAM_POP; i++)
		buffer[i] = 0;
	
	// Para cada vagas da próxima população ainda não preenchida
	for (i = (int) TAM_POP * TAX_CRUZAM + 1; i < TAM_POP; i++) {
		
		// Escolhe um indivíduos da antiga população que ainda não foram 
			// escolhidos (a fim de não gerar uma população com indivíduos 
			// idênticos.
		do {
			indice = random() % TAM_POP;
		} while (buffer[indice] == 1);
		
		// Define-o como utilizado.
		buffer[indice] = 1;
		// Referencia-o na nova população.
		(*proxima)[i] = (*atual)[indice];
	}
	
	// Assim, alguns indivíduos não serão referenciados e por isso devem ser
		// eliminados
	// Para cada item da população antiga
	for (i = 0; i < TAM_POP; i++) {
		// Verifica se o item não foi escolhido.
		if (buffer[i] == 0) {
			// Se sim, elimina-o
			free((*atual)[i]->tarefas);
			(*atual)[i]->tarefas = 0;
			(*atual)[i]->avaliacao = -1;
		}
		
		// Reseta a população antiga para se tornar um 'proxima população'
		(*atual)[i] = NULL;
	}
	
	// Comuta as populações.
	ind_buffer = (*atual);
	(*atual) = (*proxima);
	(*proxima) = ind_buffer;	
}



/*
 * Procedimento que copia os dados de um indivíduo para outro.
 */
void Copia_Melhor_Solucao(Solucao ** p, Solucao ** best){
	int i = 0;
	Solucao * best_pop_local = 0;
	
	// Inicia-se definido que o melhor é o primeiro indivíduo.
	best_pop_local = p[0];
	
	// Compara-se o primeiro com os outros de forma a escolher o melhor 
		// indivíduo de toda a população
	for (i = 1; i < TAM_POP; i++) {
		if (p[i]->avaliacao < best_pop_local->avaliacao)
			best_pop_local = p[i];
	}
	
	// Se o indivíduo best ainda não foi criado, cria-o
	if (*best == NULL) {
		// Cria indivíduo
		*best = calloc (1, sizeof(Solucao));
		(*best)->tarefas = calloc(QUANT_TAREFAS, sizeof(int));
		(*best)->excesso = calloc(QUANT_AGENTES, sizeof(int));
		
		// Copia do melhor encontrado
		(*best)->avaliacao = best_pop_local->avaliacao;
		(*best)->custo = best_pop_local->custo;
		
		// Copia os dados 'tarefa' e 'excesso'
		for (i = 0; i < QUANT_TAREFAS; i++) {
			(*best)->tarefas[i] = best_pop_local->tarefas[i];
			if (i < QUANT_AGENTES)
				(*best)->excesso[i] = best_pop_local->excesso[i];
		}
		
	} else {
		// verifica se o melhor encontrado é melhor que o indivíduo atual
		if ((*best)->avaliacao > 2 * 1000000 || 
				((best_pop_local->avaliacao < 2 * 1000000) && 
				(best_pop_local->custo < (*best)->custo))) {

			// Copia os dados 'tarefa' e 'excesso'// 
			for (i = 0; i < QUANT_TAREFAS; i++) {
				(*best)->tarefas[i] = best_pop_local->tarefas[i];
				if (i < QUANT_AGENTES)
					(*best)->excesso[i] = best_pop_local->excesso[i];
			}
			
			// Copia do melhor indivíduo
			(*best)->avaliacao = best_pop_local->avaliacao;
			(*best)->custo      = best_pop_local->custo;
		}
	}
}



/*
 * Procedimento que recombina dois indivíduos gerando um terceiro por meio de
 *	recombinação uniforme.
 */
int * Recombina(Solucao * i1, Solucao * i2) {
	int i = 0, * tarefas = 0;

	// Cria um novo vetor de tarefas
	tarefas = calloc(QUANT_TAREFAS, sizeof(int));
	
	// Recombina de forma uniforme
	for (i = 0; i < QUANT_TAREFAS; i++) {

		// Escolhe de forma uniforme sobre dois indivíduos
		if (random() % 2) 
			tarefas[i] = i1->tarefas[i];
		else
			tarefas[i] = i2->tarefas[i];
		
	}

	return tarefas;
}



/*
 * Procedimento que cria vários indivíduos aleatórios preenchendo a população.
 *	Além disso, é realizado a avaliação de cada um destes.
 */
void Cria_Nova_Populacao(Solucao *** P) {
	int i = 0, j = 0, k = 0, menor = 0;
	Solucao ** p_local = 0;

	p_local = *P;

	// Para cada item a ser criado
	for (i = 0; i < TAM_POP; i++) {
		
		// Aloca suas variáveis que armazenarão suas informações
		p_local[i]          = calloc(1, sizeof(Solucao));
		p_local[i]->excesso = calloc(QUANT_AGENTES, sizeof(int));
		p_local[i]->tarefas = calloc(QUANT_TAREFAS, sizeof(int));
		
		// Gera valores pra este indivíduo
		for (j = 0; j < QUANT_TAREFAS; j++) {
			// O primeiro indivíduo será gerado de forma gulosa e os outros
				// Serão uma mistura de Guloso com Aleatoriedade
			
			// Se não for o primeiro indivíduo, possui 66% de gerar valores
				// por meio de função randomica
			if (i > 0 && random() % 3 != 0) {
				p_local[i]->tarefas[j] = random() % QUANT_AGENTES;
				
			// Caso contrário, utiliza uma geração gulosa pra esta tarefa.
			} else {
				// O método guloso escolhe o recurso mais leve desta tarefa
				p_local[i]->tarefas[j] = 0;
				menor = 0;

				// Seleciona o agente que utiliza o menor recurso desta
					// tarefa
				for (k = 1; k < QUANT_AGENTES; k++) {
					if (RECURSOS_A_T[k][j] < RECURSOS_A_T[menor][j]) {
						menor = k;
						p_local[i]->tarefas[j] = k;
					}
				}
			}
		}

		// Avalia o novo indivíduo gerado
		Avalia_Solucao(p_local[i]);
	}
}



/*
 * Procedimento que realiza a seleção de dois indivíduos para a geração de um
 *	terceiro.
 */
void Gera_Filhos(Solucao ** selecao, Solucao *** filhos) {
	int i = 0, j = 0, numero_torneio = 0;
	Solucao * i1 = 0, * i2 = 0, * buffer = 0;
	
	Solucao ** filhos_local = * filhos;
	
	// O primeiro indivíduo gerado é o indivíduo com melhor avaliação de toda
		// população. Sendo assim, é realizado uma busca na população do 
		// melhor indivíduo e adiciona-o na próxima geração.
	j = 0;
	for (i = 1; i < TAM_POP; i++)
		if (selecao[i]->avaliacao < selecao[j]->avaliacao)
			j = i;
	
	// Aloca memória pra nova solução
	filhos_local[0]            = calloc(1, sizeof(Solucao));
	filhos_local[0]->excesso   = calloc(QUANT_AGENTES, sizeof(int));
	filhos_local[0]->tarefas   = calloc(QUANT_TAREFAS, sizeof(int));
	
	filhos_local[0]->avaliacao = selecao[j]->avaliacao;
	
	// Copia as tarefas do elemento escolhido
	for (i = 0; i < QUANT_TAREFAS; i++)
		filhos_local[0]->tarefas[i] = selecao[j]->tarefas[i];
	

	// Para os outros indivíduos que devem ser gerados, serão selecionados dois
		// pais por torneio e realizados a sua recombinação
	for (i = 1; i < (int) TAM_POP * TAX_CRUZAM + 1; i++) {
	
		// Quantidade de indivíduos que disputarão o torneio
		numero_torneio = 1 + random() % ((TAM_POP / 2) + 1);
	
		// Aloca o novo filho ainda vazio.
		filhos_local[i] = calloc(1, sizeof(Solucao));
		filhos_local[i]->excesso = calloc(QUANT_AGENTES, sizeof(int));
		
		// Realiza o torneio 1
		i1 = selecao[random() % TAM_POP];
		
		// Realiza torneio pra seleção do primeiro pai
		j = 0;
		while (j++ < numero_torneio) {
			buffer = selecao[random() % TAM_POP];
			
			if (buffer->avaliacao < i1->avaliacao)
				i1 = buffer;
		}
		
		// Realiza o torneio 2
		i2 = selecao[random() % TAM_POP];
		
		// Realiza torneio pra seleção do segundo pai
		j = 0;
		while (j++ < numero_torneio) {
			buffer = selecao[random() % TAM_POP];
			
			if (buffer->avaliacao < i2->avaliacao)
				i2 = buffer;
		}
		
		// Verifica se os dois pais são idênticos, caso sim, gera um outro
			// pai aleatório sem torneio e seleciona-o
		if (i1 == i2)
			i2 = selecao[random() % TAM_POP];
		
		// Recombina os dois indivíduos
		filhos_local[i]->tarefas = Recombina(i1, i2);
	}
}



/*
 * Procedimento que realiza a mutação dos filhos por meio do processo Creep
 *	Mutation.
 * 
 * Não são todos os filhos que são mutados. Eles são escolhidos aleatoriamente 
 *	e TAX_MUT representa a porcentagem de mutação que cada indivíduo receberá.
 * Ao final, este novo é avaliado.
 */
void Creep_Mutation(Solucao *** pop) {
	int i = 0, j = 0, indice_tarefa = 0, quant_filhos = 0, agente_atual = 0;

	// Quantidade de filhos gerados nesta próxima geração
	quant_filhos = (int) TAM_POP * TAX_CRUZAM + 1;

	// Para cada filho
	for (i = 0; i < quant_filhos; i++) {

		// Realiza mutação em uma porcentagem de do indivíduo
		for (j = 0; j < (int) QUANT_TAREFAS * TAX_MUT; j++) {

			// Muta a tarefa do indivíduo
			indice_tarefa = random() % QUANT_TAREFAS;

			agente_atual = (*pop)[i]->tarefas[indice_tarefa];

			do {
				(*pop)[i]->tarefas[indice_tarefa] = random() % QUANT_AGENTES;
			} while ((*pop)[i]->tarefas[indice_tarefa] == agente_atual);
		}	

		// Avalia o novo indivíduo gerado.
		Avalia_Solucao((*pop)[i]);
	}	
}



/*
 * Processo de geração de filhos e mutação.
 */
void Recombinacao(Solucao ** atual_p, Solucao *** proxima_p) {
	
	// Gera filhos da próxima geraçãp
	Gera_Filhos(atual_p, proxima_p);

	// Muta os filhos gerados
	Creep_Mutation(proxima_p);
}


/*
 * Algoritmo Genético segundo Lopes 2008.
 */
Solucao * Algoritmo_Genetico() {
	Solucao ** p_atual = 0, ** p_proxima = 0;
	Solucao * melhor_Solucao = 0;

	// Instancia as duas populações
	Instancia_Populacoes(&p_atual);
	Instancia_Populacoes(&p_proxima);
	
	// Gera a população inicial para execução
	Cria_Nova_Populacao(&p_atual);
	
	// Copia o melhor indivíduo gerado aleatoriamnte
	Copia_Melhor_Solucao(p_atual, &melhor_Solucao);
	
	// Inicia o contador de tempo
	start   = time(NULL);
    endwait = start + SECONDS;
	
	do {
		// Realiza a recombinação gerando novos filhos já mutados
		Recombinacao(p_atual, &p_proxima);
		
		// Completa a população com indivíduos da população anterior
		Seleciona_Nova_Geracao(&p_atual, &p_proxima);
		
		// Verifica a existência de um indivíduo melhor que o atual conhecido
		Copia_Melhor_Solucao(p_atual, &melhor_Solucao);
		
		// Recebe o tempo atual.
		start = time(NULL);
		
		// Verifica se o tempo excedeu o limite estabelecido por parâmentro.
	} while (start < endwait);

	// Desaloca populações após o fim do algoritmo
	Desaloca_Populacao(&p_atual);
	Desaloca_Populacao(&p_proxima);
	
	// Retorna o melhor indivíduo encontrado
	return melhor_Solucao;
}



/*
 * Procedimento principal
 */
int main(int argc, char** argv) {
	Solucao * solve;
	printf("\n/*");
	printf("\n * Trabalho de Projeto e Análise de Algoritmo");
 	printf("\n * Período 16.1");
 	printf("\n * ");
 	printf("\n * Desenvolver Metaheurísticas para o Problema de Alocação Generalizada");
 	printf("\n * ");
 	printf("\n * Algoritmo: Genético.");
 	printf("\n * Data: 01/08/2016.");
 	printf("\n * Distribuição Livre, desde que referenciando o autor.");
 	printf("\n * ");
 	printf("\n * Professor: Haroldo Santos");
 	printf("\n * ");
 	printf("\n * Autor do Trabalho: Rodolfo Labiapari Mansur Guimarães");
 	printf("\n */");
	
	if (argc != 4) {
		printf("\n\nErro nos parâmetros! Quantidade lida: %d\t Quantidade requerida: %d.", argc, 4);
		printf("\nnome_programa arq_configuracao arq_instancia seed\n\n");
		exit(-1);
	}
 	//printf("\n\nExecutando...\n");
	
	char * instancia = argv[2];
	char * configuracao = argv[1];
	srand(atoi(argv[3]));
	
	Le_Instancia(instancia);
	Le_Parametros(configuracao);
	
	out = fopen ("out_genetico.txt", "a");
	
	solve = Algoritmo_Genetico();
	
	Imprime_Solucao(solve);
	
	free(solve->tarefas);
	free(solve);
	
	fflush(out);
	fclose(out);
	
	fflush(stdout);

	return (EXIT_SUCCESS);
}
		\end{minted}


	\subsubsection{Algoritmo Recozimento Simulado} \label{cod:sa}

		\begin{minted} [ frame=lines, framesep=2mm, tabsize=3, breaklines=true, baselinestretch=1.2, linenos, fontsize=\footnotesize ]{c}
/*
 * Trabalho de Projeto e Análise de Algoritmo
 * Período 16.1
 * 
 * Desenvolver Metaheurísticas para o Problema de Alocação Generalizada
 * 
 * Algoritmo: Simulated Annealing.
 * Data: 01/08/2016.
 * Distribuição Livre, desde que referenciando o autor.
 * 
 * Professor: Haroldo Santos
 * 
 * Autor do Trabalho: Rodolfo Labiapari Mansur Guimarães
 */

#include <stdio.h>
#include <stdlib.h>
#include <math.h>
#include <time.h>
#include "gap.h"

int TEMPERATURA = 0;


/*
 * Procedimento que realiza a leitura dos parâmetros de configuração do 
 *	algoritmo.
 */
void Le_Parametros(char * conf) {
	FILE * f;

	f = fopen(conf, "r");

	if (f) {
		fscanf(f, "%d", &TEMPERATURA);
		fscanf(f, "%d", &MAXIteracoes);
		
		fclose(f);
		
	} else {
		printf("Erro ao ler Configuração!\n");
		exit(-1);
	}
}



/*
 * Procedimento que realiza a alteração da temperatura por meio de método 
 *	logaritmo.
 */
void Atualiza_Temperatura(double * t) {
	*t = 0.995 * *t;
}



/*
 * Procedimento de recozimento simulado, baseado no Lopes 2008.
 */
Solucao * Simulated_Annealing() {
	int iteracoes = 0;
	double fator_Boltzmann = 0, temperatura = 0, delta = 0, condicao_parada = 0, aceitacao_aleatoria = 0;
	Solucao * melhor_s = 0, * atual_s = 0, * possivel_s = 0;
	
	// Define valores iniciais
	temperatura     = TEMPERATURA;
	condicao_parada = 0.2;
	
	// Instancia soluções aleatórias para início de execução
	melhor_s   = Instancia_Solucao_Aleatoria();
	atual_s    = Instancia_Solucao_Aleatoria();
	possivel_s = Instancia_Solucao_Aleatoria();
	
	// verifica se alguma solução aleatória gerada é boa
	Teste_Aceita_Nova_Solucao(0, &melhor_s, atual_s);
	Teste_Aceita_Nova_Solucao(0, &melhor_s, possivel_s);
	
	// Enquanto tiver temperatura suficiente
	while (temperatura > condicao_parada) {
		
		// Aperfeiçoa a solução desta temperatura.
		while(iteracoes++ < MAXIteracoes) {
			// Gera um novo vizinho
			Gera_Vizinho(atual_s, &possivel_s);

			delta = possivel_s->avaliacao - atual_s->avaliacao;
			
			// verifica se a solução atual é válida
			if (delta < 0) {
				// Se sim aceita.
				Aceita_Nova_Solucao(&atual_s, possivel_s);
				
				// Verifica se é melhor que a melhor
				Teste_Aceita_Nova_Solucao(temperatura, &melhor_s, possivel_s);
				
			} else {
				// calcula fator Boltzmann
				aceitacao_aleatoria = random() / ((double)(RAND_MAX));
				fator_Boltzmann = exp(- (delta / (double) temperatura));
				
				if (aceitacao_aleatoria < fator_Boltzmann) {
					Aceita_Nova_Solucao(&atual_s, possivel_s);
				} 
			}
		}
		
		// Atualiza temperatura
		Atualiza_Temperatura(&temperatura);

		//Imprime_Status(temperatura, melhor_s);

		iteracoes = 0;
	}

	// Desaloca soluções
	Desaloca_Solucao(&atual_s);
	Desaloca_Solucao(&possivel_s);
	
	return melhor_s;
}


/*
 * Procedimento principal
 */
int main(int argc, char** argv) {
	Solucao * solve;
	
	printf("\n/*");
	printf("\n * Trabalho de Projeto e Análise de Algoritmo");
 	printf("\n * Período 16.1");
 	printf("\n * ");
 	printf("\n * Desenvolver Metaheurísticas para o Problema de Alocação Generalizada");
 	printf("\n * ");
 	printf("\n * Algoritmo: Simulated Annealing.");
 	printf("\n * Data: 01/08/2016.");
 	printf("\n * Distribuição Livre, desde que referenciando o autor.");
 	printf("\n * ");
 	printf("\n * Professor: Haroldo Santos");
 	printf("\n * ");
 	printf("\n * Autor do Trabalho: Rodolfo Labiapari Mansur Guimarães");
 	printf("\n */");
	
	if (argc != 4) {
		printf("\n\nErro nos parâmetros! Quantidade lida: %d\t Quantidade requerida: %d.", argc, 4);
		printf("\nnome_programa arq_configuracao arq_instancia seed\n\n");
		exit(-1);
	}
	
 	//printf("\n\nExecutando...\n");
	
	char * instancia = argv[2];
	char * configuracao = argv[1];
	srand(atoi(argv[3]));
	
	Le_Instancia(instancia);
	Le_Parametros(configuracao);

	out = fopen ("out_simulated_annealing.txt", "a");
	
	solve = Simulated_Annealing();

	Imprime_Solucao(solve);
	
	Desaloca_Solucao(&solve);
	fclose(out);

	return (EXIT_SUCCESS);
}
		\end{minted}


	\subsubsection{Método Reinício} \label{cod:greedy}

		\begin{minted} [ frame=lines, framesep=2mm, tabsize=3, breaklines=true, baselinestretch=1.2, linenos, fontsize=\footnotesize ]{c}

/*
 * Trabalho de Projeto e Análise de Algoritmo
 * Período 16.1
 * 
 * Desenvolver Metaheurísticas para o Problema de Alocação Generalizada
 * 
 * Algoritmo: Guloso.
 * Data: 01/08/2016.
 * Distribuição Livre, desde que referenciando o autor.
 * 
 * Professor: Haroldo Santos
 * 
 * Autor do Trabalho: Rodolfo Labiapari Mansur Guimarães
 */

#include <stdio.h>
#include <stdlib.h>
#include <math.h>
#include <limits.h>
#include <time.h>
#include "gap.h"



/*
 * Procedimento que realiza a leitura dos parâmetros de configuração do 
 *	algoritmo.
 */
void Le_Parametros(char * config) {
	FILE * f;

	f = fopen(config, "r");

	if (f) {
		fscanf(f, "%d", &SECONDS);
		fscanf(f, "%d", &MAXIteracoes);
		
		fclose(f);
		
	} else {
		printf("Erro ao ler Configuração!\n");
		exit(-1);
	}
}


/*
   Método de reinício:
      1 - Gera uma solução aleatória S
      2 - Pesquisa em uma vizinhança N(S) por uma solução melhor.
         Se a melhor solução S' pertencente a N(S) é melhor do que S, então S = S', volte para passo 2.
      3 - Atualize a melhor solução encontrada até o momento (Solução inculbente)
      4 - Se houver tempo, volta para passo 1.
 */
Solucao * Metodo_reinicio() {
	int i = 0;
	Solucao * melhor_global = 0, * atual_s = 0, * vizinha_s = 0;
	
	melhor_global  = Instancia_Solucao();
	atual_s        = Instancia_Solucao();
	vizinha_s      = Instancia_Solucao();
	
	// Inicia o contador de tempo
	start   = time(NULL);
    endwait = start + SECONDS;
	
	do {
		// Gera soluções aleatórias
		Gera_Solucao_Aleatoria(&atual_s);
		//Gera_Solucao_Aleatoria(&vizinha_s);
	
		// Testa se a solução é melhor que a atual
		Teste_Aceita_Nova_Solucao(endwait - start, &melhor_global, atual_s);
		//Teste_Aceita_Nova_Solucao(endwait - start, &melhor_global, vizinha_s);
		
		// Realiza MAXIteracoes de vizinhos a procura de soluções
			// melhoras que a atual.
		for (i = 0; i < MAXIteracoes; i++) {
			Gera_Vizinho(atual_s, &vizinha_s);
			
			if (vizinha_s->avaliacao < atual_s->avaliacao)
				Aceita_Nova_Solucao(&atual_s, vizinha_s);
		}
		
		// Verifica se na procura de soluções vizinhas, algoma é boa
		Teste_Aceita_Nova_Solucao(endwait - start, &melhor_global, atual_s);
		
		// Atualiza o tempo
		start = time(NULL);

		//Imprime_Status((double) endwait - start, melhor_global);

	} while (start < endwait);
	
	// Desaloca solução
	Desaloca_Solucao(&atual_s);
	Desaloca_Solucao(&vizinha_s);
	
	return melhor_global;
}

/*
 * Procedimento principal
 */
int main(int argc, char** argv) {
	Solucao * solve;

	printf("\n/*");
	printf("\n * Trabalho de Projeto e Análise de Algoritmo");
 	printf("\n * Período 16.1");
 	printf("\n * ");
 	printf("\n * Desenvolver Metaheurísticas para o Problema de Alocação Generalizada");
 	printf("\n * ");
 	printf("\n * Algoritmo: Guloso.");
 	printf("\n * Data: 01/08/2016.");
 	printf("\n * Distribuição Livre, desde que referenciando o autor.");
 	printf("\n * ");
 	printf("\n * Professor: Haroldo Santos");
 	printf("\n * ");
 	printf("\n * Autor do Trabalho: Rodolfo Labiapari Mansur Guimarães");
 	printf("\n */");
	
	if (argc != 4) {
		printf("\n\nErro nos parâmetros! Quantidade lida: %d\t Quantidade requerida: %d.", argc, 4);
		printf("\nnome_programa arq_configuracao arq_instancia seed\n\n");
		exit(-1);
	}
	
 	//printf("\n\nExecutando...\n");
	
	char * instancia = argv[2];
	char * configuracao = argv[1];
	srand(atoi(argv[3]));
	
	Le_Instancia(instancia);
	
	Le_Parametros(configuracao);

	out = fopen ("out_reinicio.txt", "a");
	
	solve = Metodo_reinicio();

	Imprime_Solucao(solve);
	Desaloca_Solucao(&solve);
	fclose(out);

	return (EXIT_SUCCESS);
}
		\end{minted}


	\subsubsection{Algoritmo GRASP} \label{cod:grasp}

		\begin{minted} [ frame=lines, framesep=2mm, tabsize=3, breaklines=true, baselinestretch=1.2, linenos, fontsize=\footnotesize ]{c}
/*
 * Trabalho de Projeto e Análise de Algoritmo
 * Período 16.1
 * 
 * Desenvolver Metaheurísticas para o Problema de Alocação Generalizada
 * 
 * Algoritmo: GRASP.
 * Data: 01/08/2016.
 * Distribuição Livre, desde que referenciando o autor.
 * 
 * Professor: Haroldo Santos
 * 
 * Autor do Trabalho: Rodolfo Labiapari Mansur Guimarães
 */

#include <stdio.h>
#include <stdlib.h>
#include <math.h>
#include <limits.h>
#include <time.h>
#include "gap.h"


/*
 * Procedimento que realiza a leitura dos parâmetros de configuração do 
 *	algoritmo.
 */
void Le_Parametros(char * config) {
	FILE * f;

	f = fopen(config, "r");

	if (f) {
		fscanf(f, "%d", &SECONDS);
		fscanf(f, "%d", &MAXIteracoes);
		
		fclose(f);
		
	} else {
		printf("Erro ao ler Configuração!\n");
		exit(-1);
	}
}



/*
 * Método Guloso Randômico
 */
void GreedyRandomizedConstruction(Solucao ** s, float alfa) {
	int i = 0, j = 0, min = 0, max = 0, fator = 0, sum_recursos = 0;
	char solucao_invalida = 0;

	for (i = 0; i < QUANT_AGENTES; i++)
		(*s)->excesso[i] = 0;

	(*s)->custo = 0;
	
	// Para cada tarefa
	for (i = 0; i < QUANT_TAREFAS; i++) {
		min = max = 0;

		// Encontra os valores máximos e mínimos dos agentes
		for (j = 1; j < QUANT_AGENTES; j++) {
			if (RECURSOS_A_T[j][i] + CUSTO_A_T[j][i] < RECURSOS_A_T[min][i] + CUSTO_A_T[min][i])
				min = j;
			
			if (RECURSOS_A_T[j][i] + CUSTO_A_T[j][i] > RECURSOS_A_T[max][i] + CUSTO_A_T[max][i])
				max = j;
		}
		
		// Calcula um fator de acordo com o valor alfa
		fator = RECURSOS_A_T[min][i] + alfa * (RECURSOS_A_T[max][i] - RECURSOS_A_T[min][i]);

		// procura o agente que tem maior proximidade com o fator
		min = 0;
		for (j = 1; j < QUANT_AGENTES; j++) {
			if (abs(RECURSOS_A_T[j][i] - fator) < abs(RECURSOS_A_T[min][i] - fator))
				min = j;
		}
		
		// Atribui a esta tarefa
		(*s)->tarefas[i] = min;

		// Calcula a quantidade de recusto utilizado ao atribuir a nova tarefa ao agente.
		(*s)->excesso[min] += RECURSOS_A_T[min][i];

		// Calcula o custo daquela tarefa
		(*s)->custo += CUSTO_A_T[min][i];
	}

	// Verifica se a solução gerada é válida
	for (j = 0; j < QUANT_AGENTES; j++) {
		sum_recursos += (*s)->excesso[j];

		if ((*s)->excesso[j] > CAPAC_AGENTES[j]) {
			solucao_invalida = 1;
		}
	}

	// Calcula o fator avaliação 
	if (solucao_invalida) 
		(*s)->avaliacao = ((double) sum_recursos ) * 1000000;
	else {
		(*s)->avaliacao = ((double) (*s)->custo);
	}
}
	

/*
 * Método GRASP
 */
Solucao * GRASP() {
	int i = 0;
	Solucao * melhor_global = 0, * atual_s = 0, * vizinha_s = 0;
	
	// Instancia soluções
	melhor_global  = Instancia_Solucao();
	atual_s        = Instancia_Solucao();
	vizinha_s      = Instancia_Solucao();
	
	// Inicia o contador de tempo
	start   = time(NULL);
    endwait = start + SECONDS;
	
	do {
		// Gera um novo indivíduo
		GreedyRandomizedConstruction(&atual_s, random() / RAND_MAX);
		
		// Verifica sua avaliação
		Teste_Aceita_Nova_Solucao(endwait - start, &melhor_global, atual_s);
		
		// Gera N vizinhos e verifica suas avaliações
		for (i = 0; i < MAXIteracoes; i++) {
			Gera_Vizinho(atual_s, &vizinha_s);
			
			if (vizinha_s->avaliacao < atual_s->avaliacao)
				Aceita_Nova_Solucao(&atual_s, vizinha_s);
		}
		
		// Testa se é a melhor solução encontrada
		Teste_Aceita_Nova_Solucao(endwait - start, &melhor_global, atual_s);	
		
		// Verifica se ainda possui tempo
		start = time(NULL);
	} while (start < endwait);
	
	Desaloca_Solucao(&atual_s);
	Desaloca_Solucao(&vizinha_s);

	return melhor_global;
}




/*
 * 
 */
int main(int argc, char** argv) {
	Solucao * solve;

	printf("\n/*");
	printf("\n * Trabalho de Projeto e Análise de Algoritmo");
 	printf("\n * Período 16.1");
 	printf("\n * ");
 	printf("\n * Desenvolver Metaheurísticas para o Problema de Alocação Generalizada");
 	printf("\n * ");
 	printf("\n * Algoritmo: GRASP.");
 	printf("\n * Data: 01/08/2016.");
 	printf("\n * Distribuição Livre, desde que referenciando o autor.");
 	printf("\n * ");
 	printf("\n * Professor: Haroldo Santos");
 	printf("\n * ");
 	printf("\n * Autor do Trabalho: Rodolfo Labiapari Mansur Guimarães");
 	printf("\n */");
	
	if (argc != 4) {
		printf("\n\nErro nos parâmetros! Quantidade lida: %d\t Quantidade requerida: %d.", argc, 4);
		printf("\nnome_programa arq_configuracao arq_instancia seed\n\n");
		exit(-1);
	}
 	//printf("\n\nExecutando...\n");
	
	char * instancia = argv[2];
	char * configuracao = argv[1];
	srand(atoi(argv[3]));
	
	Le_Instancia(instancia);
	
	Le_Parametros(configuracao);

	out = fopen ("out_grasp.txt", "a");
	
	solve = GRASP();

	Imprime_Solucao(solve);
	Desaloca_Solucao(&solve);
	fclose(out);

	return (EXIT_SUCCESS);
}
		\end{minted}

	
	\bibliography{biblio}
	
	
	\newpage
	\section{Anexos} \label{sec:anexo}
	
	{ \scriptsize
		\begin{longtable}{cc|c|cc|cc}
			\caption{Tabela com todos os valores obtidos.} \label{tab:resulAll} \\
			\hline
			\textbf{Algoritmo} & \textbf{Arquivo}   & \textbf{Iteração/\textit{Seed}} & \textbf{Agentes} & \textbf{Tarefas} & \textbf{Valor Encontrado} & \textbf{Valor Ótimo Literatura} \\ \hline \hline
			GA                 & \textit{a05100}    & 0                               & 05               & 100              & 1698.000000                          & 1698 \\  
			GA                 & \textit{a05100}    & 1                               & 05               & 100              & 1698.000000                          & 1698 \\  
			GA                 & \textit{a05100}    & 2                               & 05               & 100              & 1698.000000                          & 1698 \\  
			GA                 & \textit{a05100}    & 3                               & 05               & 100              & 1698.000000                          & 1698 \\  
			GA                 & \textit{a05100}    & 4                               & 05               & 100              & 1698.000000                          & 1698 \\  
			GA                 & \textit{a05100}    & 5                               & 05               & 100              & 1698.000000                          & 1698 \\  
			GA                 & \textit{a05100}    & 6                               & 05               & 100              & 1698.000000                          & 1698 \\  
			GA                 & \textit{a05100}    & 7                               & 05               & 100              & 1698.000000                          & 1698 \\  
			GA                 & \textit{a05100}    & 8                               & 05               & 100              & 1698.000000                          & 1698 \\  
			GA                 & \textit{a05100}    & 9                               & 05               & 100              & 1698.000000                          & 1698 \\  \hline
			GA                 & \textit{a10100}    & 0                               & 10               & 100              & 1360.000000                          & 1360 \\  
			GA                 & \textit{a10100}    & 1                               & 10               & 100              & 1360.000000                          & 1360 \\  
			GA                 & \textit{a10100}    & 2                               & 10               & 100              & 1360.000000                          & 1360 \\  
			GA                 & \textit{a10100}    & 3                               & 10               & 100              & 1360.000000                          & 1360 \\  
			GA                 & \textit{a10100}    & 4                               & 10               & 100              & 1360.000000                          & 1360 \\  
			GA                 & \textit{a10100}    & 5                               & 10               & 100              & 1360.000000                          & 1360 \\  
			GA                 & \textit{a10100}    & 6                               & 10               & 100              & 1360.000000                          & 1360 \\  
			GA                 & \textit{a10100}    & 7                               & 10               & 100              & 1360.000000                          & 1360 \\  
			GA                 & \textit{a10100}    & 8                               & 10               & 100              & 1360.000000                          & 1360 \\  
			GA                 & \textit{a10100}    & 9                               & 10               & 100              & 1360.000000                          & 1360 \\  \hline
			GA                 & \textit{a20100}    & 0                               & 20               & 100              & 1158.000000                          & 1158 \\  
			GA                 & \textit{a20100}    & 1                               & 20               & 100              & 1158.000000                          & 1158 \\  
			GA                 & \textit{a20100}    & 2                               & 20               & 100              & 1158.000000                          & 1158 \\  
			GA                 & \textit{a20100}    & 3                               & 20               & 100              & 1158.000000                          & 1158 \\  
			GA                 & \textit{a20100}    & 4                               & 20               & 100              & 1158.000000                          & 1158 \\  
			GA                 & \textit{a20100}    & 5                               & 20               & 100              & 1158.000000                          & 1158 \\  
			GA                 & \textit{a20100}    & 6                               & 20               & 100              & 1158.000000                          & 1158 \\  
			GA                 & \textit{a20100}    & 7                               & 20               & 100              & 1158.000000                          & 1158 \\  
			GA                 & \textit{a20100}    & 8                               & 20               & 100              & 1158.000000                          & 1158 \\  
			GA                 & \textit{a20100}    & 9                               & 20               & 100              & 1158.000000                          & 1158 \\  \hline
			GA                 & \textit{a05200}    & 0                               & 05               & 200              & 3235.000000                          & 3235 \\  
			GA                 & \textit{a05200}    & 1                               & 05               & 200              & 3235.000000                          & 3235 \\  
			GA                 & \textit{a05200}    & 2                               & 05               & 200              & 3235.000000                          & 3235 \\  
			GA                 & \textit{a05200}    & 3                               & 05               & 200              & 3235.000000                          & 3235 \\  
			GA                 & \textit{a05200}    & 4                               & 05               & 200              & 3235.000000                          & 3235 \\  
			GA                 & \textit{a05200}    & 5                               & 05               & 200              & 3235.000000                          & 3235 \\  
			GA                 & \textit{a05200}    & 6                               & 05               & 200              & 3235.000000                          & 3235 \\  
			GA                 & \textit{a05200}    & 7                               & 05               & 200              & 3235.000000                          & 3235 \\  
			GA                 & \textit{a05200}    & 8                               & 05               & 200              & 3235.000000                          & 3235 \\  
			GA                 & \textit{a05200}    & 9                               & 05               & 200              & 3235.000000                          & 3235 \\  \hline
			GA                 & \textit{a10200}    & 0                               & 10               & 200              & 2623.000000                          & 2623 \\  
			GA                 & \textit{a10200}    & 1                               & 10               & 200              & 2623.000000                          & 2623 \\  
			GA                 & \textit{a10200}    & 2                               & 10               & 200              & 2623.000000                          & 2623 \\  
			GA                 & \textit{a10200}    & 3                               & 10               & 200              & 2623.000000                          & 2623 \\  
			GA                 & \textit{a10200}    & 4                               & 10               & 200              & 2623.000000                          & 2623 \\  
			GA                 & \textit{a10200}    & 5                               & 10               & 200              & 2623.000000                          & 2623 \\  
			GA                 & \textit{a10200}    & 6                               & 10               & 200              & 2623.000000                          & 2623 \\  
			GA                 & \textit{a10200}    & 7                               & 10               & 200              & 2623.000000                          & 2623 \\  
			GA                 & \textit{a10200}    & 8                               & 10               & 200              & 2623.000000                          & 2623 \\  
			GA                 & \textit{a10200}    & 9                               & 10               & 200              & 2623.000000                          & 2623 \\  \hline
			GA                 & \textit{a20200}    & 0                               & 20               & 200              & 2339.000000                          & 2339 \\  
			GA                 & \textit{a20200}    & 1                               & 20               & 200              & 2339.000000                          & 2339 \\  
			GA                 & \textit{a20200}    & 2                               & 20               & 200              & 2339.000000                          & 2339 \\  
			GA                 & \textit{a20200}    & 3                               & 20               & 200              & 2339.000000                          & 2339 \\  
			GA                 & \textit{a20200}    & 4                               & 20               & 200              & 2339.000000                          & 2339 \\  
			GA                 & \textit{a20200}    & 5                               & 20               & 200              & 2339.000000                          & 2339 \\  
			GA                 & \textit{a20200}    & 6                               & 20               & 200              & 2339.000000                          & 2339 \\  
			GA                 & \textit{a20200}    & 7                               & 20               & 200              & 2339.000000                          & 2339 \\  
			GA                 & \textit{a20200}    & 8                               & 20               & 200              & 2339.000000                          & 2339 \\  
			GA                 & \textit{a20200}    & 9                               & 20               & 200              & 2339.000000                          & 2339 \\  \hline
			GA                 & \textit{c05100}    & 0                               & 05               & 100              & 1982.000000                          & 1931 \\  
			GA                 & \textit{c05100}    & 1                               & 05               & 100              & 1982.000000                          & 1931 \\  
			GA                 & \textit{c05100}    & 2                               & 05               & 100              & 1982.000000                          & 1931 \\  
			GA                 & \textit{c05100}    & 3                               & 05               & 100              & 1982.000000                          & 1931 \\  
			GA                 & \textit{c05100}    & 4                               & 05               & 100              & 1982.000000                          & 1931 \\  
			GA                 & \textit{c05100}    & 5                               & 05               & 100              & 1982.000000                          & 1931 \\  
			GA                 & \textit{c05100}    & 6                               & 05               & 100              & 1982.000000                          & 1931 \\  
			GA                 & \textit{c05100}    & 7                               & 05               & 100              & 1982.000000                          & 1931 \\  
			GA                 & \textit{c05100}    & 8                               & 05               & 100              & 1982.000000                          & 1931 \\  
			GA                 & \textit{c05100}    & 9                               & 05               & 100              & 1982.000000                          & 1931 \\  \hline
			GA                 & \textit{c10100}    & 0                               & 10               & 100              & 1439.000000                          & 1402 \\  
			GA                 & \textit{c10100}    & 1                               & 10               & 100              & 1439.000000                          & 1402 \\  
			GA                 & \textit{c10100}    & 2                               & 10               & 100              & 1439.000000                          & 1402 \\  
			GA                 & \textit{c10100}    & 3                               & 10               & 100              & 1439.000000                          & 1402 \\  
			GA                 & \textit{c10100}    & 4                               & 10               & 100              & 1439.000000                          & 1402 \\  
			GA                 & \textit{c10100}    & 5                               & 10               & 100              & 1439.000000                          & 1402 \\  
			GA                 & \textit{c10100}    & 6                               & 10               & 100              & 1439.000000                          & 1402 \\  
			GA                 & \textit{c10100}    & 7                               & 10               & 100              & 1439.000000                          & 1402 \\  
			GA                 & \textit{c10100}    & 8                               & 10               & 100              & 1439.000000                          & 1402 \\  
			GA                 & \textit{c10100}    & 9                               & 10               & 100              & 1439.000000                          & 1402 \\  \hline
			GA                 & \textit{c20100}    & 0                               & 20               & 100              & 1281.000000                          & 1243 \\  
			GA                 & \textit{c20100}    & 1                               & 20               & 100              & 1281.000000                          & 1243 \\  
			GA                 & \textit{c20100}    & 2                               & 20               & 100              & 1281.000000                          & 1243 \\  
			GA                 & \textit{c20100}    & 3                               & 20               & 100              & 1281.000000                          & 1243 \\  
			GA                 & \textit{c20100}    & 4                               & 20               & 100              & 1281.000000                          & 1243 \\  
			GA                 & \textit{c20100}    & 5                               & 20               & 100              & 1281.000000                          & 1243 \\  
			GA                 & \textit{c20100}    & 6                               & 20               & 100              & 1281.000000                          & 1243 \\  
			GA                 & \textit{c20100}    & 7                               & 20               & 100              & 1281.000000                          & 1243 \\  
			GA                 & \textit{c20100}    & 8                               & 20               & 100              & 1282.000000                          & 1243 \\  
			GA                 & \textit{c20100}    & 9                               & 20               & 100              & 1282.000000                          & 1243 \\  \hline
			GA                 & \textit{c05200}    & 0                               & 05               & 200              & 3552.000000                          & 3456 \\  
			GA                 & \textit{c05200}    & 1                               & 05               & 200              & 3554.000000                          & 3456 \\  
			GA                 & \textit{c05200}    & 2                               & 05               & 200              & 3552.000000                          & 3456 \\  
			GA                 & \textit{c05200}    & 3                               & 05               & 200              & 3554.000000                          & 3456 \\  
			GA                 & \textit{c05200}    & 4                               & 05               & 200              & 3552.000000                          & 3456 \\  
			GA                 & \textit{c05200}    & 5                               & 05               & 200              & 3552.000000                          & 3456 \\  
			GA                 & \textit{c05200}    & 6                               & 05               & 200              & 3552.000000                          & 3456 \\  
			GA                 & \textit{c05200}    & 7                               & 05               & 200              & 3552.000000                          & 3456 \\  
			GA                 & \textit{c05200}    & 8                               & 05               & 200              & 3552.000000                          & 3456 \\  
			GA                 & \textit{c05200}    & 9                               & 05               & 200              & 3554.000000                          & 3456 \\  \hline
			GA                 & \textit{c10200}    & 0                               & 10               & 200              & 3006.000000                          & 2806 \\  
			GA                 & \textit{c10200}    & 1                               & 10               & 200              & 3006.000000                          & 2806 \\  
			GA                 & \textit{c10200}    & 2                               & 10               & 200              & 3006.000000                          & 2806 \\  
			GA                 & \textit{c10200}    & 3                               & 10               & 200              & 3006.000000                          & 2806 \\  
			GA                 & \textit{c10200}    & 4                               & 10               & 200              & 3006.000000                          & 2806 \\  
			GA                 & \textit{c10200}    & 5                               & 10               & 200              & 3006.000000                          & 2806 \\  
			GA                 & \textit{c10200}    & 6                               & 10               & 200              & 3006.000000                          & 2806 \\  
			GA                 & \textit{c10200}    & 7                               & 10               & 200              & 3006.000000                          & 2806 \\  
			GA                 & \textit{c10200}    & 8                               & 10               & 200              & 3006.000000                          & 2806 \\  
			GA                 & \textit{c10200}    & 9                               & 10               & 200              & 3006.000000                          & 2806 \\  \hline
			GA                 & \textit{c20200}    & 0                               & 20               & 200              & 2590.000000                          & 2391 \\  
			GA                 & \textit{c20200}    & 1                               & 20               & 200              & 2588.000000                          & 2391 \\  
			GA                 & \textit{c20200}    & 2                               & 20               & 200              & 2586.000000                          & 2391 \\  
			GA                 & \textit{c20200}    & 3                               & 20               & 200              & 2586.000000                          & 2391 \\  
			GA                 & \textit{c20200}    & 4                               & 20               & 200              & 2588.000000                          & 2391 \\  
			GA                 & \textit{c20200}    & 5                               & 20               & 200              & 2588.000000                          & 2391 \\  
			GA                 & \textit{c20200}    & 6                               & 20               & 200              & 2586.000000                          & 2391 \\  
			GA                 & \textit{c20200}    & 7                               & 20               & 200              & 2586.000000                          & 2391 \\  
			GA                 & \textit{c20200}    & 8                               & 20               & 200              & 2586.000000                          & 2391 \\  
			GA                 & \textit{c20200}    & 9                               & 20               & 200              & 2586.000000                          & 2391 \\  \hline
			GA                 & \textit{e05100}    & 0                               & 05               & 100              & 13895.000000                          & 1267 \\ 
			GA                 & \textit{e05100}    & 1                               & 05               & 100              & 13895.000000                          & 1267 \\ 
			GA                 & \textit{e05100}    & 2                               & 05               & 100              & 13895.000000                          & 1267 \\ 
			GA                 & \textit{e05100}    & 3                               & 05               & 100              & 13895.000000                          & 1267 \\ 
			GA                 & \textit{e05100}    & 4                               & 05               & 100              & 13895.000000                          & 1267 \\ 
			GA                 & \textit{e05100}    & 5                               & 05               & 100              & 13895.000000                          & 1267 \\ 
			GA                 & \textit{e05100}    & 6                               & 05               & 100              & 13895.000000                          & 1267 \\ 
			GA                 & \textit{e05100}    & 7                               & 05               & 100              & 13895.000000                          & 1267 \\ 
			GA                 & \textit{e05100}    & 8                               & 05               & 100              & 13895.000000                          & 1267 \\ 
			GA                 & \textit{e05100}    & 9                               & 05               & 100              & 13895.000000                          & 1267 \\ \hline
			GA                 & \textit{e10100}    & 0                               & 10               & 100              & 13918.000000                          & 1156 \\ 
			GA                 & \textit{e10100}    & 1                               & 10               & 100              & 13918.000000                          & 1156 \\ 
			GA                 & \textit{e10100}    & 2                               & 10               & 100              & 13918.000000                          & 1156 \\ 
			GA                 & \textit{e10100}    & 3                               & 10               & 100              & 13918.000000                          & 1156 \\ 
			GA                 & \textit{e10100}    & 4                               & 10               & 100              & 13918.000000                          & 1156 \\ 
			GA                 & \textit{e10100}    & 5                               & 10               & 100              & 13918.000000                          & 1156 \\ 
			GA                 & \textit{e10100}    & 6                               & 10               & 100              & 13918.000000                          & 1156 \\ 
			GA                 & \textit{e10100}    & 7                               & 10               & 100              & 13918.000000                          & 1156 \\ 
			GA                 & \textit{e10100}    & 8                               & 10               & 100              & 13918.000000                          & 1156 \\ 
			GA                 & \textit{e10100}    & 9                               & 10               & 100              & 13918.000000                          & 1156 \\ \hline
			GA                 & \textit{e20100}    & 0                               & 20               & 100              & 10168.000000                          & 8431 \\ 
			GA                 & \textit{e20100}    & 1                               & 20               & 100              & 10168.000000                          & 8431 \\ 
			GA                 & \textit{e20100}    & 2                               & 20               & 100              & 10168.000000                          & 8431 \\ 
			GA                 & \textit{e20100}    & 3                               & 20               & 100              & 10168.000000                          & 8431 \\ 
			GA                 & \textit{e20100}    & 4                               & 20               & 100              & 10168.000000                          & 8431 \\ 
			GA                 & \textit{e20100}    & 5                               & 20               & 100              & 10168.000000                          & 8431 \\ 
			GA                 & \textit{e20100}    & 6                               & 20               & 100              & 10168.000000                          & 8431 \\ 
			GA                 & \textit{e20100}    & 7                               & 20               & 100              & 10168.000000                          & 8431 \\ 
			GA                 & \textit{e20100}    & 8                               & 20               & 100              & 10168.000000                          & 8431 \\ 
			GA                 & \textit{e20100}    & 9                               & 20               & 100              & 10168.000000                          & 8431 \\ \hline
			GA                 & \textit{e05200}    & 0                               & 05               & 200              & 26819.000000                          & 2492 \\ 
			GA                 & \textit{e05200}    & 1                               & 05               & 200              & 26819.000000                          & 2492 \\ 
			GA                 & \textit{e05200}    & 2                               & 05               & 200              & 26819.000000                          & 2492 \\ 
			GA                 & \textit{e05200}    & 3                               & 05               & 200              & 26819.000000                          & 2492 \\ 
			GA                 & \textit{e05200}    & 4                               & 05               & 200              & 26819.000000                          & 2492 \\ 
			GA                 & \textit{e05200}    & 5                               & 05               & 200              & 26819.000000                          & 2492 \\ 
			GA                 & \textit{e05200}    & 6                               & 05               & 200              & 26819.000000                          & 2492 \\ 
			GA                 & \textit{e05200}    & 7                               & 05               & 200              & 26819.000000                          & 2492 \\ 
			GA                 & \textit{e05200}    & 8                               & 05               & 200              & 26819.000000                          & 2492 \\ 
			GA                 & \textit{e05200}    & 9                               & 05               & 200              & 26819.000000                          & 2492 \\ \hline
			GA                 & \textit{e10200}    & 0                               & 10               & 200              & 27188.000000                          & 2330 \\ 
			GA                 & \textit{e10200}    & 1                               & 10               & 200              & 27188.000000                          & 2330 \\ 
			GA                 & \textit{e10200}    & 2                               & 10               & 200              & 27188.000000                          & 2330 \\ 
			GA                 & \textit{e10200}    & 3                               & 10               & 200              & 27188.000000                          & 2330 \\ 
			GA                 & \textit{e10200}    & 4                               & 10               & 200              & 27188.000000                          & 2330 \\ 
			GA                 & \textit{e10200}    & 5                               & 10               & 200              & 27188.000000                          & 2330 \\ 
			GA                 & \textit{e10200}    & 6                               & 10               & 200              & 27188.000000                          & 2330 \\ 
			GA                 & \textit{e10200}    & 7                               & 10               & 200              & 27188.000000                          & 2330 \\ 
			GA                 & \textit{e10200}    & 8                               & 10               & 200              & 27188.000000                          & 2330 \\ 
			GA                 & \textit{e10200}    & 9                               & 10               & 200              & 27188.000000                          & 2330 \\ \hline
			GA                 & \textit{e20200}    & 0                               & 20               & 200              & 27583.000000                          & 2237 \\ 
			GA                 & \textit{e20200}    & 1                               & 20               & 200              & 27583.000000                          & 2237 \\ 
			GA                 & \textit{e20200}    & 2                               & 20               & 200              & 27583.000000                          & 2237 \\ 
			GA                 & \textit{e20200}    & 3                               & 20               & 200              & 27583.000000                          & 2237 \\ 
			GA                 & \textit{e20200}    & 4                               & 20               & 200              & 27583.000000                          & 2237 \\ 
			GA                 & \textit{e20200}    & 5                               & 20               & 200              & 27583.000000                          & 2237 \\ 
			GA                 & \textit{e20200}    & 6                               & 20               & 200              & 27583.000000                          & 2237 \\ 
			GA                 & \textit{e20200}    & 7                               & 20               & 200              & 27583.000000                          & 2237 \\ 
			GA                 & \textit{e20200}    & 8                               & 20               & 200              & 27583.000000                          & 2237 \\ 
			GA                 & \textit{e20200}    & 9                               & 20               & 200              & 27583.000000                          & 2237 \\ \hline
			\hline \hline
			\textbf{Algoritmo} & \textbf{Arquivo}   & \textbf{Iteração/\textit{Seed}} & \textbf{Agentes} & \textbf{Tarefas} & \textbf{Valor Encontrado} & \textbf{Valor Ótimo Literatura} \\ \hline
			Recozimento Simulado & \textit{a05100}    & 0                               & 05               & 100              & 1698.000000                          & 1698 \\ 
			Recozimento Simulado & \textit{a05100}    & 1                               & 05               & 100              & 1698.000000                          & 1698 \\ 
			Recozimento Simulado & \textit{a05100}    & 2                               & 05               & 100              & 1698.000000                          & 1698 \\ 
			Recozimento Simulado & \textit{a05100}    & 3                               & 05               & 100              & 1698.000000                          & 1698 \\ 
			Recozimento Simulado & \textit{a05100}    & 4                               & 05               & 100              & 1698.000000                          & 1698 \\ 
			Recozimento Simulado & \textit{a05100}    & 5                               & 05               & 100              & 1698.000000                          & 1698 \\ 
			Recozimento Simulado & \textit{a05100}    & 6                               & 05               & 100              & 1698.000000                          & 1698 \\ 
			Recozimento Simulado & \textit{a05100}    & 7                               & 05               & 100              & 1698.000000                          & 1698 \\ 
			Recozimento Simulado & \textit{a05100}    & 8                               & 05               & 100              & 1698.000000                          & 1698 \\ 
			Recozimento Simulado & \textit{a05100}    & 9                               & 05               & 100              & 1698.000000                          & 1698 \\ \hline
			Recozimento Simulado & \textit{a10100}    & 0                               & 10               & 100              & 1360.000000                          & 1360 \\ 
			Recozimento Simulado & \textit{a10100}    & 1                               & 10               & 100              & 1360.000000                          & 1360 \\ 
			Recozimento Simulado & \textit{a10100}    & 2                               & 10               & 100              & 1360.000000                          & 1360 \\ 
			Recozimento Simulado & \textit{a10100}    & 3                               & 10               & 100              & 1360.000000                          & 1360 \\ 
			Recozimento Simulado & \textit{a10100}    & 4                               & 10               & 100              & 1360.000000                          & 1360 \\ 
			Recozimento Simulado & \textit{a10100}    & 5                               & 10               & 100              & 1360.000000                          & 1360 \\ 
			Recozimento Simulado & \textit{a10100}    & 6                               & 10               & 100              & 1360.000000                          & 1360 \\ 
			Recozimento Simulado & \textit{a10100}    & 7                               & 10               & 100              & 1360.000000                          & 1360 \\ 
			Recozimento Simulado & \textit{a10100}    & 8                               & 10               & 100              & 1360.000000                          & 1360 \\ 
			Recozimento Simulado & \textit{a10100}    & 9                               & 10               & 100              & 1360.000000                          & 1360 \\ \hline
			Recozimento Simulado & \textit{a20100}    & 0                               & 20               & 100              & 1158.000000                          & 1158 \\ 
			Recozimento Simulado & \textit{a20100}    & 1                               & 20               & 100              & 1158.000000                          & 1158 \\ 
			Recozimento Simulado & \textit{a20100}    & 2                               & 20               & 100              & 1158.000000                          & 1158 \\ 
			Recozimento Simulado & \textit{a20100}    & 3                               & 20               & 100              & 1158.000000                          & 1158 \\ 
			Recozimento Simulado & \textit{a20100}    & 4                               & 20               & 100              & 1158.000000                          & 1158 \\ 
			Recozimento Simulado & \textit{a20100}    & 5                               & 20               & 100              & 1158.000000                          & 1158 \\ 
			Recozimento Simulado & \textit{a20100}    & 6                               & 20               & 100              & 1158.000000                          & 1158 \\ 
			Recozimento Simulado & \textit{a20100}    & 7                               & 20               & 100              & 1158.000000                          & 1158 \\ 
			Recozimento Simulado & \textit{a20100}    & 8                               & 20               & 100              & 1158.000000                          & 1158 \\ 
			Recozimento Simulado & \textit{a20100}    & 9                               & 20               & 100              & 1158.000000                          & 1158 \\ \hline
			Recozimento Simulado & \textit{a05200}    & 0                               & 05               & 200              & 3235.000000                          & 3235 \\ 
			Recozimento Simulado & \textit{a05200}    & 1                               & 05               & 200              & 3235.000000                          & 3235 \\ 
			Recozimento Simulado & \textit{a05200}    & 2                               & 05               & 200              & 3235.000000                          & 3235 \\ 
			Recozimento Simulado & \textit{a05200}    & 3                               & 05               & 200              & 3235.000000                          & 3235 \\ 
			Recozimento Simulado & \textit{a05200}    & 4                               & 05               & 200              & 3235.000000                          & 3235 \\ 
			Recozimento Simulado & \textit{a05200}    & 5                               & 05               & 200              & 3235.000000                          & 3235 \\ 
			Recozimento Simulado & \textit{a05200}    & 6                               & 05               & 200              & 3235.000000                          & 3235 \\ 
			Recozimento Simulado & \textit{a05200}    & 7                               & 05               & 200              & 3235.000000                          & 3235 \\ 
			Recozimento Simulado & \textit{a05200}    & 8                               & 05               & 200              & 3235.000000                          & 3235 \\ 
			Recozimento Simulado & \textit{a05200}    & 9                               & 05               & 200              & 3235.000000                          & 3235 \\ \hline
			Recozimento Simulado & \textit{a10200}    & 0                               & 10               & 200              & 2623.000000                          & 2623 \\ 
			Recozimento Simulado & \textit{a10200}    & 1                               & 10               & 200              & 2623.000000                          & 2623 \\ 
			Recozimento Simulado & \textit{a10200}    & 2                               & 10               & 200              & 2623.000000                          & 2623 \\ 
			Recozimento Simulado & \textit{a10200}    & 3                               & 10               & 200              & 2623.000000                          & 2623 \\ 
			Recozimento Simulado & \textit{a10200}    & 4                               & 10               & 200              & 2623.000000                          & 2623 \\ 
			Recozimento Simulado & \textit{a10200}    & 5                               & 10               & 200              & 2623.000000                          & 2623 \\ 
			Recozimento Simulado & \textit{a10200}    & 6                               & 10               & 200              & 2623.000000                          & 2623 \\ 
			Recozimento Simulado & \textit{a10200}    & 7                               & 10               & 200              & 2623.000000                          & 2623 \\ 
			Recozimento Simulado & \textit{a10200}    & 8                               & 10               & 200              & 2623.000000                          & 2623 \\ 
			Recozimento Simulado & \textit{a10200}    & 9                               & 10               & 200              & 2623.000000                          & 2623 \\ \hline
			Recozimento Simulado & \textit{a20200}    & 0                               & 20               & 200              & 2339.000000                          & 2339 \\ 
			Recozimento Simulado & \textit{a20200}    & 1                               & 20               & 200              & 2339.000000                          & 2339 \\ 
			Recozimento Simulado & \textit{a20200}    & 2                               & 20               & 200              & 2339.000000                          & 2339 \\ 
			Recozimento Simulado & \textit{a20200}    & 3                               & 20               & 200              & 2339.000000                          & 2339 \\ 
			Recozimento Simulado & \textit{a20200}    & 4                               & 20               & 200              & 2339.000000                          & 2339 \\ 
			Recozimento Simulado & \textit{a20200}    & 5                               & 20               & 200              & 2339.000000                          & 2339 \\ 
			Recozimento Simulado & \textit{a20200}    & 6                               & 20               & 200              & 2339.000000                          & 2339 \\ 
			Recozimento Simulado & \textit{a20200}    & 7                               & 20               & 200              & 2339.000000                          & 2339 \\ 
			Recozimento Simulado & \textit{a20200}    & 8                               & 20               & 200              & 2339.000000                          & 2339 \\ 
			Recozimento Simulado & \textit{a20200}    & 9                               & 20               & 200              & 2339.000000                          & 2339 \\ \hline
			Recozimento Simulado & \textit{c05100}    & 0                               & 05               & 100              & 1937.000000                          & 1931 \\ 
			Recozimento Simulado & \textit{c05100}    & 1                               & 05               & 100              & 1937.000000                          & 1931 \\ 
			Recozimento Simulado & \textit{c05100}    & 2                               & 05               & 100              & 1937.000000                          & 1931 \\ 
			Recozimento Simulado & \textit{c05100}    & 3                               & 05               & 100              & 1937.000000                          & 1931 \\ 
			Recozimento Simulado & \textit{c05100}    & 4                               & 05               & 100              & 1937.000000                          & 1931 \\ 
			Recozimento Simulado & \textit{c05100}    & 5                               & 05               & 100              & 1937.000000                          & 1931 \\ 
			Recozimento Simulado & \textit{c05100}    & 6                               & 05               & 100              & 1937.000000                          & 1931 \\ 
			Recozimento Simulado & \textit{c05100}    & 7                               & 05               & 100              & 1937.000000                          & 1931 \\ 
			Recozimento Simulado & \textit{c05100}    & 8                               & 05               & 100              & 1937.000000                          & 1931 \\ 
			Recozimento Simulado & \textit{c05100}    & 9                               & 05               & 100              & 1937.000000                          & 1931 \\ \hline
			Recozimento Simulado & \textit{c10100}    & 0                               & 10               & 100              & 1415.000000                          & 1402 \\ 
			Recozimento Simulado & \textit{c10100}    & 1                               & 10               & 100              & 1415.000000                          & 1402 \\ 
			Recozimento Simulado & \textit{c10100}    & 2                               & 10               & 100              & 1415.000000                          & 1402 \\ 
			Recozimento Simulado & \textit{c10100}    & 3                               & 10               & 100              & 1415.000000                          & 1402 \\ 
			Recozimento Simulado & \textit{c10100}    & 4                               & 10               & 100              & 1415.000000                          & 1402 \\ 
			Recozimento Simulado & \textit{c10100}    & 5                               & 10               & 100              & 1415.000000                          & 1402 \\ 
			Recozimento Simulado & \textit{c10100}    & 6                               & 10               & 100              & 1415.000000                          & 1402 \\ 
			Recozimento Simulado & \textit{c10100}    & 7                               & 10               & 100              & 1415.000000                          & 1402 \\ 
			Recozimento Simulado & \textit{c10100}    & 8                               & 10               & 100              & 1415.000000                          & 1402 \\ 
			Recozimento Simulado & \textit{c10100}    & 9                               & 10               & 100              & 1415.000000                          & 1402 \\ \hline
			Recozimento Simulado & \textit{c20100}    & 0                               & 20               & 100              & 1264.000000                          & 1243 \\ 
			Recozimento Simulado & \textit{c20100}    & 1                               & 20               & 100              & 1264.000000                          & 1243 \\ 
			Recozimento Simulado & \textit{c20100}    & 2                               & 20               & 100              & 1264.000000                          & 1243 \\ 
			Recozimento Simulado & \textit{c20100}    & 3                               & 20               & 100              & 1264.000000                          & 1243 \\ 
			Recozimento Simulado & \textit{c20100}    & 4                               & 20               & 100              & 1264.000000                          & 1243 \\ 
			Recozimento Simulado & \textit{c20100}    & 5                               & 20               & 100              & 1264.000000                          & 1243 \\ 
			Recozimento Simulado & \textit{c20100}    & 6                               & 20               & 100              & 1264.000000                          & 1243 \\ 
			Recozimento Simulado & \textit{c20100}    & 7                               & 20               & 100              & 1264.000000                          & 1243 \\ 
			Recozimento Simulado & \textit{c20100}    & 8                               & 20               & 100              & 1264.000000                          & 1243 \\ 
			Recozimento Simulado & \textit{c20100}    & 9                               & 20               & 100              & 1264.000000                          & 1243 \\ \hline
			Recozimento Simulado & \textit{c05200}    & 0                               & 05               & 200              & 3460.000000                          & 3456 \\ 
			Recozimento Simulado & \textit{c05200}    & 1                               & 05               & 200              & 3460.000000                          & 3456 \\ 
			Recozimento Simulado & \textit{c05200}    & 2                               & 05               & 200              & 3460.000000                          & 3456 \\ 
			Recozimento Simulado & \textit{c05200}    & 3                               & 05               & 200              & 3460.000000                          & 3456 \\ 
			Recozimento Simulado & \textit{c05200}    & 4                               & 05               & 200              & 3460.000000                          & 3456 \\ 
			Recozimento Simulado & \textit{c05200}    & 5                               & 05               & 200              & 3460.000000                          & 3456 \\ 
			Recozimento Simulado & \textit{c05200}    & 6                               & 05               & 200              & 3460.000000                          & 3456 \\ 
			Recozimento Simulado & \textit{c05200}    & 7                               & 05               & 200              & 3460.000000                          & 3456 \\ 
			Recozimento Simulado & \textit{c05200}    & 8                               & 05               & 200              & 3460.000000                          & 3456 \\ 
			Recozimento Simulado & \textit{c05200}    & 9                               & 05               & 200              & 3460.000000                          & 3456 \\ \hline
			Recozimento Simulado & \textit{c10200}    & 0                               & 10               & 200              & 2838.000000                          & 2806 \\ 
			Recozimento Simulado & \textit{c10200}    & 1                               & 10               & 200              & 2838.000000                          & 2806 \\ 
			Recozimento Simulado & \textit{c10200}    & 2                               & 10               & 200              & 2838.000000                          & 2806 \\ 
			Recozimento Simulado & \textit{c10200}    & 3                               & 10               & 200              & 2838.000000                          & 2806 \\ 
			Recozimento Simulado & \textit{c10200}    & 4                               & 10               & 200              & 2838.000000                          & 2806 \\ 
			Recozimento Simulado & \textit{c10200}    & 5                               & 10               & 200              & 2838.000000                          & 2806 \\ 
			Recozimento Simulado & \textit{c10200}    & 6                               & 10               & 200              & 2838.000000                          & 2806 \\ 
			Recozimento Simulado & \textit{c10200}    & 7                               & 10               & 200              & 2838.000000                          & 2806 \\ 
			Recozimento Simulado & \textit{c10200}    & 8                               & 10               & 200              & 2838.000000                          & 2806 \\ 
			Recozimento Simulado & \textit{c10200}    & 9                               & 10               & 200              & 2838.000000                          & 2806 \\ \hline
			Recozimento Simulado & \textit{c20200}    & 0                               & 20               & 200              & 2413.000000                          & 2391 \\ 
			Recozimento Simulado & \textit{c20200}    & 1                               & 20               & 200              & 2413.000000                          & 2391 \\ 
			Recozimento Simulado & \textit{c20200}    & 2                               & 20               & 200              & 2413.000000                          & 2391 \\ 
			Recozimento Simulado & \textit{c20200}    & 3                               & 20               & 200              & 2413.000000                          & 2391 \\ 
			Recozimento Simulado & \textit{c20200}    & 4                               & 20               & 200              & 2413.000000                          & 2391 \\ 
			Recozimento Simulado & \textit{c20200}    & 5                               & 20               & 200              & 2413.000000                          & 2391 \\ 
			Recozimento Simulado & \textit{c20200}    & 6                               & 20               & 200              & 2413.000000                          & 2391 \\ 
			Recozimento Simulado & \textit{c20200}    & 7                               & 20               & 200              & 2413.000000                          & 2391 \\ 
			Recozimento Simulado & \textit{c20200}    & 8                               & 20               & 200              & 2413.000000                          & 2391 \\ 
			Recozimento Simulado & \textit{c20200}    & 9                               & 20               & 200              & 2413.000000                          & 2391 \\ \hline
			Recozimento Simulado & \textit{e05100}    & 0                               & 05               & 100              & 12762.000000                          & 1267 \\ 
			Recozimento Simulado & \textit{e05100}    & 1                               & 05               & 100              & 12762.000000                          & 1267 \\ 
			Recozimento Simulado & \textit{e05100}    & 2                               & 05               & 100              & 12762.000000                          & 1267 \\ 
			Recozimento Simulado & \textit{e05100}    & 3                               & 05               & 100              & 12762.000000                          & 1267 \\ 
			Recozimento Simulado & \textit{e05100}    & 4                               & 05               & 100              & 12762.000000                          & 1267 \\ 
			Recozimento Simulado & \textit{e05100}    & 5                               & 05               & 100              & 12762.000000                          & 1267 \\ 
			Recozimento Simulado & \textit{e05100}    & 6                               & 05               & 100              & 12762.000000                          & 1267 \\ 
			Recozimento Simulado & \textit{e05100}    & 7                               & 05               & 100              & 12762.000000                          & 1267 \\ 
			Recozimento Simulado & \textit{e05100}    & 8                               & 05               & 100              & 12762.000000                          & 1267 \\ 
			Recozimento Simulado & \textit{e05100}    & 9                               & 05               & 100              & 12762.000000                          & 1267 \\ \hline
			Recozimento Simulado & \textit{e10100}    & 0                               & 10               & 100              & 11833.000000                          & 1156 \\ 
			Recozimento Simulado & \textit{e10100}    & 1                               & 10               & 100              & 11833.000000                          & 1156 \\ 
			Recozimento Simulado & \textit{e10100}    & 2                               & 10               & 100              & 11833.000000                          & 1156 \\ 
			Recozimento Simulado & \textit{e10100}    & 3                               & 10               & 100              & 11833.000000                          & 1156 \\ 
			Recozimento Simulado & \textit{e10100}    & 4                               & 10               & 100              & 11833.000000                          & 1156 \\ 
			Recozimento Simulado & \textit{e10100}    & 5                               & 10               & 100              & 11833.000000                          & 1156 \\ 
			Recozimento Simulado & \textit{e10100}    & 6                               & 10               & 100              & 11833.000000                          & 1156 \\ 
			Recozimento Simulado & \textit{e10100}    & 7                               & 10               & 100              & 11833.000000                          & 1156 \\ 
			Recozimento Simulado & \textit{e10100}    & 8                               & 10               & 100              & 11833.000000                          & 1156 \\ 
			Recozimento Simulado & \textit{e10100}    & 9                               & 10               & 100              & 11833.000000                          & 1156 \\ \hline
			Recozimento Simulado & \textit{e20100}    & 0                               & 20               & 100              & 8837.000000                          & 8431 \\ 
			Recozimento Simulado & \textit{e20100}    & 1                               & 20               & 100              & 8837.000000                          & 8431 \\ 
			Recozimento Simulado & \textit{e20100}    & 2                               & 20               & 100              & 8837.000000                          & 8431 \\ 
			Recozimento Simulado & \textit{e20100}    & 3                               & 20               & 100              & 8837.000000                          & 8431 \\ 
			Recozimento Simulado & \textit{e20100}    & 4                               & 20               & 100              & 8837.000000                          & 8431 \\ 
			Recozimento Simulado & \textit{e20100}    & 5                               & 20               & 100              & 8837.000000                          & 8431 \\ 
			Recozimento Simulado & \textit{e20100}    & 6                               & 20               & 100              & 8837.000000                          & 8431 \\ 
			Recozimento Simulado & \textit{e20100}    & 7                               & 20               & 100              & 8837.000000                          & 8431 \\ 
			Recozimento Simulado & \textit{e20100}    & 8                               & 20               & 100              & 8837.000000                          & 8431 \\ 
			Recozimento Simulado & \textit{e20100}    & 9                               & 20               & 100              & 8837.000000                          & 8431 \\ \hline
			Recozimento Simulado & \textit{e05200}    & 0                               & 05               & 200              & 25111.000000                          & 2492 \\ 
			Recozimento Simulado & \textit{e05200}    & 1                               & 05               & 200              & 25111.000000                          & 2492 \\ 
			Recozimento Simulado & \textit{e05200}    & 2                               & 05               & 200              & 25111.000000                          & 2492 \\ 
			Recozimento Simulado & \textit{e05200}    & 3                               & 05               & 200              & 25111.000000                          & 2492 \\ 
			Recozimento Simulado & \textit{e05200}    & 4                               & 05               & 200              & 25111.000000                          & 2492 \\ 
			Recozimento Simulado & \textit{e05200}    & 5                               & 05               & 200              & 25111.000000                          & 2492 \\ 
			Recozimento Simulado & \textit{e05200}    & 6                               & 05               & 200              & 25111.000000                          & 2492 \\ 
			Recozimento Simulado & \textit{e05200}    & 7                               & 05               & 200              & 25111.000000                          & 2492 \\ 
			Recozimento Simulado & \textit{e05200}    & 8                               & 05               & 200              & 25111.000000                          & 2492 \\ 
			Recozimento Simulado & \textit{e05200}    & 9                               & 05               & 200              & 25111.000000                          & 2492 \\ \hline
			Recozimento Simulado & \textit{e10200}    & 0                               & 10               & 200              & 23825.000000                          & 2330 \\ 
			Recozimento Simulado & \textit{e10200}    & 1                               & 10               & 200              & 23825.000000                          & 2330 \\ 
			Recozimento Simulado & \textit{e10200}    & 2                               & 10               & 200              & 23825.000000                          & 2330 \\ 
			Recozimento Simulado & \textit{e10200}    & 3                               & 10               & 200              & 23825.000000                          & 2330 \\ 
			Recozimento Simulado & \textit{e10200}    & 4                               & 10               & 200              & 23825.000000                          & 2330 \\ 
			Recozimento Simulado & \textit{e10200}    & 5                               & 10               & 200              & 23825.000000                          & 2330 \\ 
			Recozimento Simulado & \textit{e10200}    & 6                               & 10               & 200              & 23825.000000                          & 2330 \\ 
			Recozimento Simulado & \textit{e10200}    & 7                               & 10               & 200              & 23825.000000                          & 2330 \\ 
			Recozimento Simulado & \textit{e10200}    & 8                               & 10               & 200              & 23825.000000                          & 2330 \\ 
			Recozimento Simulado & \textit{e10200}    & 9                               & 10               & 200              & 23825.000000                          & 2330 \\ \hline
			Recozimento Simulado & \textit{e20200}    & 0                               & 20               & 200              & 23571.000000                          & 2237 \\ 
			Recozimento Simulado & \textit{e20200}    & 1                               & 20               & 200              & 23571.000000                          & 2237 \\ 
			Recozimento Simulado & \textit{e20200}    & 2                               & 20               & 200              & 23571.000000                          & 2237 \\ 
			Recozimento Simulado & \textit{e20200}    & 3                               & 20               & 200              & 23571.000000                          & 2237 \\ 
			Recozimento Simulado & \textit{e20200}    & 4                               & 20               & 200              & 23571.000000                          & 2237 \\ 
			Recozimento Simulado & \textit{e20200}    & 5                               & 20               & 200              & 23571.000000                          & 2237 \\ 
			Recozimento Simulado & \textit{e20200}    & 6                               & 20               & 200              & 23571.000000                          & 2237 \\ 
			Recozimento Simulado & \textit{e20200}    & 7                               & 20               & 200              & 23571.000000                          & 2237 \\ 
			Recozimento Simulado & \textit{e20200}    & 8                               & 20               & 200              & 23571.000000                          & 2237 \\ 
			Recozimento Simulado & \textit{e20200}    & 9                               & 20               & 200              & 23571.000000                          & 2237 \\ \hline
			\hline \hline
			\textbf{Algoritmo} & \textbf{Arquivo}   & \textbf{Iteração/\textit{Seed}} & \textbf{Agentes} & \textbf{Tarefas} & \textbf{Valor Encontrado} & \textbf{Valor Ótimo Literatura} \\ \hline
			Método Reinício    & \textit{a05100}    & 0                               & 05               & 100              & 1698.000000                          & 1698 \\ 
			Método Reinício    & \textit{a05100}    & 1                               & 05               & 100              & 1698.000000                          & 1698 \\ 
			Método Reinício    & \textit{a05100}    & 2                               & 05               & 100              & 1698.000000                          & 1698 \\ 
			Método Reinício    & \textit{a05100}    & 3                               & 05               & 100              & 1698.000000                          & 1698 \\ 
			Método Reinício    & \textit{a05100}    & 4                               & 05               & 100              & 1698.000000                          & 1698 \\ 
			Método Reinício    & \textit{a05100}    & 5                               & 05               & 100              & 1698.000000                          & 1698 \\ 
			Método Reinício    & \textit{a05100}    & 6                               & 05               & 100              & 1698.000000                          & 1698 \\ 
			Método Reinício    & \textit{a05100}    & 7                               & 05               & 100              & 1698.000000                          & 1698 \\ 
			Método Reinício    & \textit{a05100}    & 8                               & 05               & 100              & 1698.000000                          & 1698 \\ 
			Método Reinício    & \textit{a05100}    & 9                               & 05               & 100              & 1698.000000                          & 1698 \\ \hline
			Método Reinício    & \textit{a10100}    & 0                               & 10               & 100              & 1360.000000                          & 1360 \\ 
			Método Reinício    & \textit{a10100}    & 1                               & 10               & 100              & 1360.000000                          & 1360 \\ 
			Método Reinício    & \textit{a10100}    & 2                               & 10               & 100              & 1360.000000                          & 1360 \\ 
			Método Reinício    & \textit{a10100}    & 3                               & 10               & 100              & 1360.000000                          & 1360 \\ 
			Método Reinício    & \textit{a10100}    & 4                               & 10               & 100              & 1360.000000                          & 1360 \\ 
			Método Reinício    & \textit{a10100}    & 5                               & 10               & 100              & 1360.000000                          & 1360 \\ 
			Método Reinício    & \textit{a10100}    & 6                               & 10               & 100              & 1360.000000                          & 1360 \\ 
			Método Reinício    & \textit{a10100}    & 7                               & 10               & 100              & 1360.000000                          & 1360 \\ 
			Método Reinício    & \textit{a10100}    & 8                               & 10               & 100              & 1360.000000                          & 1360 \\ 
			Método Reinício    & \textit{a10100}    & 9                               & 10               & 100              & 1360.000000                          & 1360 \\ \hline
			Método Reinício    & \textit{a20100}    & 0                               & 20               & 100              & 1158.000000                          & 1158 \\ 
			Método Reinício    & \textit{a20100}    & 1                               & 20               & 100              & 1158.000000                          & 1158 \\ 
			Método Reinício    & \textit{a20100}    & 2                               & 20               & 100              & 1158.000000                          & 1158 \\ 
			Método Reinício    & \textit{a20100}    & 3                               & 20               & 100              & 1158.000000                          & 1158 \\ 
			Método Reinício    & \textit{a20100}    & 4                               & 20               & 100              & 1158.000000                          & 1158 \\ 
			Método Reinício    & \textit{a20100}    & 5                               & 20               & 100              & 1158.000000                          & 1158 \\ 
			Método Reinício    & \textit{a20100}    & 6                               & 20               & 100              & 1158.000000                          & 1158 \\ 
			Método Reinício    & \textit{a20100}    & 7                               & 20               & 100              & 1158.000000                          & 1158 \\ 
			Método Reinício    & \textit{a20100}    & 8                               & 20               & 100              & 1158.000000                          & 1158 \\ 
			Método Reinício    & \textit{a20100}    & 9                               & 20               & 100              & 1158.000000                          & 1158 \\ \hline
			Método Reinício    & \textit{a05200}    & 0                               & 05               & 200              & 3235.000000                          & 3235 \\ 
			Método Reinício    & \textit{a05200}    & 1                               & 05               & 200              & 3235.000000                          & 3235 \\ 
			Método Reinício    & \textit{a05200}    & 2                               & 05               & 200              & 3235.000000                          & 3235 \\ 
			Método Reinício    & \textit{a05200}    & 3                               & 05               & 200              & 3235.000000                          & 3235 \\ 
			Método Reinício    & \textit{a05200}    & 4                               & 05               & 200              & 3235.000000                          & 3235 \\ 
			Método Reinício    & \textit{a05200}    & 5                               & 05               & 200              & 3235.000000                          & 3235 \\ 
			Método Reinício    & \textit{a05200}    & 6                               & 05               & 200              & 3235.000000                          & 3235 \\ 
			Método Reinício    & \textit{a05200}    & 7                               & 05               & 200              & 3235.000000                          & 3235 \\ 
			Método Reinício    & \textit{a05200}    & 8                               & 05               & 200              & 3235.000000                          & 3235 \\ 
			Método Reinício    & \textit{a05200}    & 9                               & 05               & 200              & 3235.000000                          & 3235 \\ \hline
			Método Reinício    & \textit{a10200}    & 0                               & 10               & 200              & 2623.000000                          & 2623 \\ 
			Método Reinício    & \textit{a10200}    & 1                               & 10               & 200              & 2623.000000                          & 2623 \\ 
			Método Reinício    & \textit{a10200}    & 2                               & 10               & 200              & 2623.000000                          & 2623 \\ 
			Método Reinício    & \textit{a10200}    & 3                               & 10               & 200              & 2623.000000                          & 2623 \\ 
			Método Reinício    & \textit{a10200}    & 4                               & 10               & 200              & 2623.000000                          & 2623 \\ 
			Método Reinício    & \textit{a10200}    & 5                               & 10               & 200              & 2623.000000                          & 2623 \\ 
			Método Reinício    & \textit{a10200}    & 6                               & 10               & 200              & 2623.000000                          & 2623 \\ 
			Método Reinício    & \textit{a10200}    & 7                               & 10               & 200              & 2623.000000                          & 2623 \\ 
			Método Reinício    & \textit{a10200}    & 8                               & 10               & 200              & 2623.000000                          & 2623 \\ 
			Método Reinício    & \textit{a10200}    & 9                               & 10               & 200              & 2623.000000                          & 2623 \\ \hline
			Método Reinício    & \textit{a20200}    & 0                               & 20               & 200              & 2339.000000                          & 2339 \\ 
			Método Reinício    & \textit{a20200}    & 1                               & 20               & 200              & 2339.000000                          & 2339 \\ 
			Método Reinício    & \textit{a20200}    & 2                               & 20               & 200              & 2339.000000                          & 2339 \\ 
			Método Reinício    & \textit{a20200}    & 3                               & 20               & 200              & 2339.000000                          & 2339 \\ 
			Método Reinício    & \textit{a20200}    & 4                               & 20               & 200              & 2339.000000                          & 2339 \\ 
			Método Reinício    & \textit{a20200}    & 5                               & 20               & 200              & 2339.000000                          & 2339 \\ 
			Método Reinício    & \textit{a20200}    & 6                               & 20               & 200              & 2339.000000                          & 2339 \\ 
			Método Reinício    & \textit{a20200}    & 7                               & 20               & 200              & 2339.000000                          & 2339 \\ 
			Método Reinício    & \textit{a20200}    & 8                               & 20               & 200              & 2339.000000                          & 2339 \\ 
			Método Reinício    & \textit{a20200}    & 9                               & 20               & 200              & 2339.000000                          & 2339 \\ \hline
			Método Reinício    & \textit{c05100}    & 0                               & 05               & 100              & 1953.000000                          & 1931 \\ 
			Método Reinício    & \textit{c05100}    & 1                               & 05               & 100              & 1953.000000                          & 1931 \\ 
			Método Reinício    & \textit{c05100}    & 2                               & 05               & 100              & 1953.000000                          & 1931 \\ 
			Método Reinício    & \textit{c05100}    & 3                               & 05               & 100              & 1953.000000                          & 1931 \\ 
			Método Reinício    & \textit{c05100}    & 4                               & 05               & 100              & 1953.000000                          & 1931 \\ 
			Método Reinício    & \textit{c05100}    & 5                               & 05               & 100              & 1953.000000                          & 1931 \\ 
			Método Reinício    & \textit{c05100}    & 6                               & 05               & 100              & 1953.000000                          & 1931 \\ 
			Método Reinício    & \textit{c05100}    & 7                               & 05               & 100              & 1953.000000                          & 1931 \\ 
			Método Reinício    & \textit{c05100}    & 8                               & 05               & 100              & 1953.000000                          & 1931 \\ 
			Método Reinício    & \textit{c05100}    & 9                               & 05               & 100              & 1953.000000                          & 1931 \\ \hline
			Método Reinício    & \textit{c10100}    & 0                               & 10               & 100              & 1433.000000                          & 1402 \\ 
			Método Reinício    & \textit{c10100}    & 1                               & 10               & 100              & 1433.000000                          & 1402 \\ 
			Método Reinício    & \textit{c10100}    & 2                               & 10               & 100              & 1433.000000                          & 1402 \\ 
			Método Reinício    & \textit{c10100}    & 3                               & 10               & 100              & 1433.000000                          & 1402 \\ 
			Método Reinício    & \textit{c10100}    & 4                               & 10               & 100              & 1433.000000                          & 1402 \\ 
			Método Reinício    & \textit{c10100}    & 5                               & 10               & 100              & 1433.000000                          & 1402 \\ 
			Método Reinício    & \textit{c10100}    & 6                               & 10               & 100              & 1433.000000                          & 1402 \\ 
			Método Reinício    & \textit{c10100}    & 7                               & 10               & 100              & 1433.000000                          & 1402 \\ 
			Método Reinício    & \textit{c10100}    & 8                               & 10               & 100              & 1433.000000                          & 1402 \\ 
			Método Reinício    & \textit{c10100}    & 9                               & 10               & 100              & 1433.000000                          & 1402 \\ \hline
			Método Reinício    & \textit{c20100}    & 0                               & 20               & 100              & 1264.000000                          & 1243 \\ 
			Método Reinício    & \textit{c20100}    & 1                               & 20               & 100              & 1264.000000                          & 1243 \\ 
			Método Reinício    & \textit{c20100}    & 2                               & 20               & 100              & 1264.000000                          & 1243 \\ 
			Método Reinício    & \textit{c20100}    & 3                               & 20               & 100              & 1264.000000                          & 1243 \\ 
			Método Reinício    & \textit{c20100}    & 4                               & 20               & 100              & 1264.000000                          & 1243 \\ 
			Método Reinício    & \textit{c20100}    & 5                               & 20               & 100              & 1264.000000                          & 1243 \\ 
			Método Reinício    & \textit{c20100}    & 6                               & 20               & 100              & 1264.000000                          & 1243 \\ 
			Método Reinício    & \textit{c20100}    & 7                               & 20               & 100              & 1264.000000                          & 1243 \\ 
			Método Reinício    & \textit{c20100}    & 8                               & 20               & 100              & 1264.000000                          & 1243 \\ 
			Método Reinício    & \textit{c20100}    & 9                               & 20               & 100              & 1264.000000                          & 1243 \\ \hline
			Método Reinício    & \textit{c05200}    & 0                               & 05               & 200              & 3503.000000                          & 3456 \\ 
			Método Reinício    & \textit{c05200}    & 1                               & 05               & 200              & 3503.000000                          & 3456 \\ 
			Método Reinício    & \textit{c05200}    & 2                               & 05               & 200              & 3503.000000                          & 3456 \\ 
			Método Reinício    & \textit{c05200}    & 3                               & 05               & 200              & 3503.000000                          & 3456 \\ 
			Método Reinício    & \textit{c05200}    & 4                               & 05               & 200              & 3503.000000                          & 3456 \\ 
			Método Reinício    & \textit{c05200}    & 5                               & 05               & 200              & 3503.000000                          & 3456 \\ 
			Método Reinício    & \textit{c05200}    & 6                               & 05               & 200              & 3503.000000                          & 3456 \\ 
			Método Reinício    & \textit{c05200}    & 7                               & 05               & 200              & 3503.000000                          & 3456 \\ 
			Método Reinício    & \textit{c05200}    & 8                               & 05               & 200              & 3503.000000                          & 3456 \\ 
			Método Reinício    & \textit{c05200}    & 9                               & 05               & 200              & 3503.000000                          & 3456 \\ \hline
			Método Reinício    & \textit{c10200}    & 0                               & 10               & 200              & 2852.000000                          & 2806 \\ 
			Método Reinício    & \textit{c10200}    & 1                               & 10               & 200              & 2852.000000                          & 2806 \\ 
			Método Reinício    & \textit{c10200}    & 2                               & 10               & 200              & 2852.000000                          & 2806 \\ 
			Método Reinício    & \textit{c10200}    & 3                               & 10               & 200              & 2852.000000                          & 2806 \\ 
			Método Reinício    & \textit{c10200}    & 4                               & 10               & 200              & 2852.000000                          & 2806 \\ 
			Método Reinício    & \textit{c10200}    & 5                               & 10               & 200              & 2852.000000                          & 2806 \\ 
			Método Reinício    & \textit{c10200}    & 6                               & 10               & 200              & 2852.000000                          & 2806 \\ 
			Método Reinício    & \textit{c10200}    & 7                               & 10               & 200              & 2852.000000                          & 2806 \\ 
			Método Reinício    & \textit{c10200}    & 8                               & 10               & 200              & 2852.000000                          & 2806 \\ 
			Método Reinício    & \textit{c10200}    & 9                               & 10               & 200              & 2852.000000                          & 2806 \\ \hline
			Método Reinício    & \textit{c20200}    & 0                               & 20               & 200              & 2445.000000                          & 2391 \\ 
			Método Reinício    & \textit{c20200}    & 1                               & 20               & 200              & 2445.000000                          & 2391 \\ 
			Método Reinício    & \textit{c20200}    & 2                               & 20               & 200              & 2445.000000                          & 2391 \\ 
			Método Reinício    & \textit{c20200}    & 3                               & 20               & 200              & 2445.000000                          & 2391 \\ 
			Método Reinício    & \textit{c20200}    & 4                               & 20               & 200              & 2445.000000                          & 2391 \\ 
			Método Reinício    & \textit{c20200}    & 5                               & 20               & 200              & 2445.000000                          & 2391 \\ 
			Método Reinício    & \textit{c20200}    & 6                               & 20               & 200              & 2445.000000                          & 2391 \\ 
			Método Reinício    & \textit{c20200}    & 7                               & 20               & 200              & 2445.000000                          & 2391 \\ 
			Método Reinício    & \textit{c20200}    & 8                               & 20               & 200              & 2445.000000                          & 2391 \\ 
			Método Reinício    & \textit{c20200}    & 9                               & 20               & 200              & 2445.000000                          & 2391 \\ \hline
			Método Reinício    & \textit{e05100}    & 0                               & 05               & 100              & 13948.000000                          & 1267 \\ 
			Método Reinício    & \textit{e05100}    & 1                               & 05               & 100              & 13948.000000                          & 1267 \\ 
			Método Reinício    & \textit{e05100}    & 2                               & 05               & 100              & 13948.000000                          & 1267 \\ 
			Método Reinício    & \textit{e05100}    & 3                               & 05               & 100              & 13948.000000                          & 1267 \\ 
			Método Reinício    & \textit{e05100}    & 4                               & 05               & 100              & 13948.000000                          & 1267 \\ 
			Método Reinício    & \textit{e05100}    & 5                               & 05               & 100              & 13948.000000                          & 1267 \\ 
			Método Reinício    & \textit{e05100}    & 6                               & 05               & 100              & 13948.000000                          & 1267 \\ 
			Método Reinício    & \textit{e05100}    & 7                               & 05               & 100              & 13948.000000                          & 1267 \\ 
			Método Reinício    & \textit{e05100}    & 8                               & 05               & 100              & 13948.000000                          & 1267 \\ 
			Método Reinício    & \textit{e05100}    & 9                               & 05               & 100              & 13948.000000                          & 1267 \\ \hline
			Método Reinício    & \textit{e10100}    & 0                               & 10               & 100              & 13199.000000                          & 1156 \\ 
			Método Reinício    & \textit{e10100}    & 1                               & 10               & 100              & 13199.000000                          & 1156 \\ 
			Método Reinício    & \textit{e10100}    & 2                               & 10               & 100              & 13199.000000                          & 1156 \\ 
			Método Reinício    & \textit{e10100}    & 3                               & 10               & 100              & 13199.000000                          & 1156 \\ 
			Método Reinício    & \textit{e10100}    & 4                               & 10               & 100              & 13199.000000                          & 1156 \\ 
			Método Reinício    & \textit{e10100}    & 5                               & 10               & 100              & 13199.000000                          & 1156 \\ 
			Método Reinício    & \textit{e10100}    & 6                               & 10               & 100              & 13199.000000                          & 1156 \\ 
			Método Reinício    & \textit{e10100}    & 7                               & 10               & 100              & 13199.000000                          & 1156 \\ 
			Método Reinício    & \textit{e10100}    & 8                               & 10               & 100              & 13199.000000                          & 1156 \\ 
			Método Reinício    & \textit{e10100}    & 9                               & 10               & 100              & 13199.000000                          & 1156 \\ \hline
			Método Reinício    & \textit{e20100}    & 0                               & 20               & 100              & 9534.000000                          & 8431 \\ 
			Método Reinício    & \textit{e20100}    & 1                               & 20               & 100              & 9534.000000                          & 8431 \\ 
			Método Reinício    & \textit{e20100}    & 2                               & 20               & 100              & 9534.000000                          & 8431 \\ 
			Método Reinício    & \textit{e20100}    & 3                               & 20               & 100              & 9534.000000                          & 8431 \\ 
			Método Reinício    & \textit{e20100}    & 4                               & 20               & 100              & 9534.000000                          & 8431 \\ 
			Método Reinício    & \textit{e20100}    & 5                               & 20               & 100              & 9534.000000                          & 8431 \\ 
			Método Reinício    & \textit{e20100}    & 6                               & 20               & 100              & 9534.000000                          & 8431 \\ 
			Método Reinício    & \textit{e20100}    & 7                               & 20               & 100              & 9534.000000                          & 8431 \\ 
			Método Reinício    & \textit{e20100}    & 8                               & 20               & 100              & 9534.000000                          & 8431 \\ 
			Método Reinício    & \textit{e20100}    & 9                               & 20               & 100              & 9534.000000                          & 8431 \\ \hline
			Método Reinício    & \textit{e05200}    & 0                               & 05               & 200              & 28693.000000                          & 2492 \\ 
			Método Reinício    & \textit{e05200}    & 1                               & 05               & 200              & 28693.000000                          & 2492 \\ 
			Método Reinício    & \textit{e05200}    & 2                               & 05               & 200              & 28693.000000                          & 2492 \\ 
			Método Reinício    & \textit{e05200}    & 3                               & 05               & 200              & 28693.000000                          & 2492 \\ 
			Método Reinício    & \textit{e05200}    & 4                               & 05               & 200              & 28693.000000                          & 2492 \\ 
			Método Reinício    & \textit{e05200}    & 5                               & 05               & 200              & 28693.000000                          & 2492 \\ 
			Método Reinício    & \textit{e05200}    & 6                               & 05               & 200              & 28693.000000                          & 2492 \\ 
			Método Reinício    & \textit{e05200}    & 7                               & 05               & 200              & 28693.000000                          & 2492 \\ 
			Método Reinício    & \textit{e05200}    & 8                               & 05               & 200              & 28693.000000                          & 2492 \\ 
			Método Reinício    & \textit{e05200}    & 9                               & 05               & 200              & 28693.000000                          & 2492 \\ \hline
			Método Reinício    & \textit{e10200}    & 0                               & 10               & 200              & 27227.000000                          & 2330 \\ 
			Método Reinício    & \textit{e10200}    & 1                               & 10               & 200              & 27227.000000                          & 2330 \\ 
			Método Reinício    & \textit{e10200}    & 2                               & 10               & 200              & 27227.000000                          & 2330 \\ 
			Método Reinício    & \textit{e10200}    & 3                               & 10               & 200              & 27227.000000                          & 2330 \\ 
			Método Reinício    & \textit{e10200}    & 4                               & 10               & 200              & 27227.000000                          & 2330 \\ 
			Método Reinício    & \textit{e10200}    & 5                               & 10               & 200              & 27227.000000                          & 2330 \\ 
			Método Reinício    & \textit{e10200}    & 6                               & 10               & 200              & 27227.000000                          & 2330 \\ 
			Método Reinício    & \textit{e10200}    & 7                               & 10               & 200              & 27227.000000                          & 2330 \\ 
			Método Reinício    & \textit{e10200}    & 8                               & 10               & 200              & 27227.000000                          & 2330 \\ 
			Método Reinício    & \textit{e10200}    & 9                               & 10               & 200              & 27227.000000                          & 2330 \\ \hline
			Método Reinício    & \textit{e20200}    & 0                               & 20               & 200              & 25992.000000                          & 2237 \\ 
			Método Reinício    & \textit{e20200}    & 1                               & 20               & 200              & 25992.000000                          & 2237 \\ 
			Método Reinício    & \textit{e20200}    & 2                               & 20               & 200              & 25992.000000                          & 2237 \\ 
			Método Reinício    & \textit{e20200}    & 3                               & 20               & 200              & 25992.000000                          & 2237 \\ 
			Método Reinício    & \textit{e20200}    & 4                               & 20               & 200              & 25992.000000                          & 2237 \\ 
			Método Reinício    & \textit{e20200}    & 5                               & 20               & 200              & 25992.000000                          & 2237 \\ 
			Método Reinício    & \textit{e20200}    & 6                               & 20               & 200              & 25992.000000                          & 2237 \\ 
			Método Reinício    & \textit{e20200}    & 7                               & 20               & 200              & 25992.000000                          & 2237 \\ 
			Método Reinício    & \textit{e20200}    & 8                               & 20               & 200              & 25992.000000                          & 2237 \\ 
			Método Reinício    & \textit{e20200}    & 9                               & 20               & 200              & 25992.000000                          & 2237 \\ \hline
			\hline \hline
			\textbf{Algoritmo} & \textbf{Arquivo}   & \textbf{Iteração/\textit{Seed}} & \textbf{Agentes} & \textbf{Tarefas} & \textbf{Valor Encontrado} & \textbf{Valor Ótimo Literatura} \\ \hline
			GRASP              & \textit{a05100}    & 0                               & 05               & 100              & 1698.000000                          & 1698 \\ 
			GRASP              & \textit{a05100}    & 1                               & 05               & 100              & 1698.000000                          & 1698 \\ 
			GRASP              & \textit{a05100}    & 2                               & 05               & 100              & 1698.000000                          & 1698 \\ 
			GRASP              & \textit{a05100}    & 3                               & 05               & 100              & 1698.000000                          & 1698 \\ 
			GRASP              & \textit{a05100}    & 4                               & 05               & 100              & 1698.000000                          & 1698 \\ 
			GRASP              & \textit{a05100}    & 5                               & 05               & 100              & 1698.000000                          & 1698 \\ 
			GRASP              & \textit{a05100}    & 6                               & 05               & 100              & 1698.000000                          & 1698 \\ 
			GRASP              & \textit{a05100}    & 7                               & 05               & 100              & 1698.000000                          & 1698 \\ 
			GRASP              & \textit{a05100}    & 8                               & 05               & 100              & 1698.000000                          & 1698 \\ 
			GRASP              & \textit{a05100}    & 9                               & 05               & 100              & 1698.000000                          & 1698 \\ \hline
			GRASP              & \textit{a10100}    & 0                               & 10               & 100              & 1360.000000                          & 1360 \\ 
			GRASP              & \textit{a10100}    & 1                               & 10               & 100              & 1360.000000                          & 1360 \\ 
			GRASP              & \textit{a10100}    & 2                               & 10               & 100              & 1360.000000                          & 1360 \\ 
			GRASP              & \textit{a10100}    & 3                               & 10               & 100              & 1360.000000                          & 1360 \\ 
			GRASP              & \textit{a10100}    & 4                               & 10               & 100              & 1360.000000                          & 1360 \\ 
			GRASP              & \textit{a10100}    & 5                               & 10               & 100              & 1360.000000                          & 1360 \\ 
			GRASP              & \textit{a10100}    & 6                               & 10               & 100              & 1360.000000                          & 1360 \\ 
			GRASP              & \textit{a10100}    & 7                               & 10               & 100              & 1360.000000                          & 1360 \\ 
			GRASP              & \textit{a10100}    & 8                               & 10               & 100              & 1360.000000                          & 1360 \\ 
			GRASP              & \textit{a10100}    & 9                               & 10               & 100              & 1360.000000                          & 1360 \\ \hline
			GRASP              & \textit{a20100}    & 0                               & 20               & 100              & 1158.000000                          & 1158 \\ 
			GRASP              & \textit{a20100}    & 1                               & 20               & 100              & 1158.000000                          & 1158 \\ 
			GRASP              & \textit{a20100}    & 2                               & 20               & 100              & 1158.000000                          & 1158 \\ 
			GRASP              & \textit{a20100}    & 3                               & 20               & 100              & 1158.000000                          & 1158 \\ 
			GRASP              & \textit{a20100}    & 4                               & 20               & 100              & 1158.000000                          & 1158 \\ 
			GRASP              & \textit{a20100}    & 5                               & 20               & 100              & 1158.000000                          & 1158 \\ 
			GRASP              & \textit{a20100}    & 6                               & 20               & 100              & 1158.000000                          & 1158 \\ 
			GRASP              & \textit{a20100}    & 7                               & 20               & 100              & 1158.000000                          & 1158 \\ 
			GRASP              & \textit{a20100}    & 8                               & 20               & 100              & 1158.000000                          & 1158 \\ 
			GRASP              & \textit{a20100}    & 9                               & 20               & 100              & 1158.000000                          & 1158 \\ \hline
			GRASP              & \textit{a05200}    & 0                               & 05               & 200              & 3235.000000                          & 3235 \\ 
			GRASP              & \textit{a05200}    & 1                               & 05               & 200              & 3235.000000                          & 3235 \\ 
			GRASP              & \textit{a05200}    & 2                               & 05               & 200              & 3235.000000                          & 3235 \\ 
			GRASP              & \textit{a05200}    & 3                               & 05               & 200              & 3235.000000                          & 3235 \\ 
			GRASP              & \textit{a05200}    & 4                               & 05               & 200              & 3235.000000                          & 3235 \\ 
			GRASP              & \textit{a05200}    & 5                               & 05               & 200              & 3235.000000                          & 3235 \\ 
			GRASP              & \textit{a05200}    & 6                               & 05               & 200              & 3235.000000                          & 3235 \\ 
			GRASP              & \textit{a05200}    & 7                               & 05               & 200              & 3235.000000                          & 3235 \\ 
			GRASP              & \textit{a05200}    & 8                               & 05               & 200              & 3235.000000                          & 3235 \\ 
			GRASP              & \textit{a05200}    & 9                               & 05               & 200              & 3235.000000                          & 3235 \\ \hline
			GRASP              & \textit{a10200}    & 0                               & 10               & 200              & 2623.000000                          & 2623 \\ 
			GRASP              & \textit{a10200}    & 1                               & 10               & 200              & 2623.000000                          & 2623 \\ 
			GRASP              & \textit{a10200}    & 2                               & 10               & 200              & 2623.000000                          & 2623 \\ 
			GRASP              & \textit{a10200}    & 3                               & 10               & 200              & 2623.000000                          & 2623 \\ 
			GRASP              & \textit{a10200}    & 4                               & 10               & 200              & 2623.000000                          & 2623 \\ 
			GRASP              & \textit{a10200}    & 5                               & 10               & 200              & 2623.000000                          & 2623 \\ 
			GRASP              & \textit{a10200}    & 6                               & 10               & 200              & 2623.000000                          & 2623 \\ 
			GRASP              & \textit{a10200}    & 7                               & 10               & 200              & 2623.000000                          & 2623 \\ 
			GRASP              & \textit{a10200}    & 8                               & 10               & 200              & 2623.000000                          & 2623 \\ 
			GRASP              & \textit{a10200}    & 9                               & 10               & 200              & 2623.000000                          & 2623 \\ \hline
			GRASP              & \textit{a20200}    & 0                               & 20               & 200              & 2339.000000                          & 2339 \\ 
			GRASP              & \textit{a20200}    & 1                               & 20               & 200              & 2339.000000                          & 2339 \\ 
			GRASP              & \textit{a20200}    & 2                               & 20               & 200              & 2339.000000                          & 2339 \\ 
			GRASP              & \textit{a20200}    & 3                               & 20               & 200              & 2339.000000                          & 2339 \\ 
			GRASP              & \textit{a20200}    & 4                               & 20               & 200              & 2339.000000                          & 2339 \\ 
			GRASP              & \textit{a20200}    & 5                               & 20               & 200              & 2339.000000                          & 2339 \\ 
			GRASP              & \textit{a20200}    & 6                               & 20               & 200              & 2339.000000                          & 2339 \\ 
			GRASP              & \textit{a20200}    & 7                               & 20               & 200              & 2339.000000                          & 2339 \\ 
			GRASP              & \textit{a20200}    & 8                               & 20               & 200              & 2339.000000                          & 2339 \\ 
			GRASP              & \textit{a20200}    & 9                               & 20               & 200              & 2339.000000                          & 2339 \\ \hline
			GRASP              & \textit{c05100}    & 0                               & 05               & 100              & 1955.000000                          & 1931 \\ 
			GRASP              & \textit{c05100}    & 1                               & 05               & 100              & 1955.000000                          & 1931 \\ 
			GRASP              & \textit{c05100}    & 2                               & 05               & 100              & 1955.000000                          & 1931 \\ 
			GRASP              & \textit{c05100}    & 3                               & 05               & 100              & 1955.000000                          & 1931 \\ 
			GRASP              & \textit{c05100}    & 4                               & 05               & 100              & 1955.000000                          & 1931 \\ 
			GRASP              & \textit{c05100}    & 5                               & 05               & 100              & 1955.000000                          & 1931 \\ 
			GRASP              & \textit{c05100}    & 6                               & 05               & 100              & 1955.000000                          & 1931 \\ 
			GRASP              & \textit{c05100}    & 7                               & 05               & 100              & 1955.000000                          & 1931 \\ 
			GRASP              & \textit{c05100}    & 8                               & 05               & 100              & 1955.000000                          & 1931 \\ 
			GRASP              & \textit{c05100}    & 9                               & 05               & 100              & 1955.000000                          & 1931 \\ \hline
			GRASP              & \textit{c10100}    & 0                               & 10               & 100              & 1419.000000                          & 1402 \\ 
			GRASP              & \textit{c10100}    & 1                               & 10               & 100              & 1419.000000                          & 1402 \\ 
			GRASP              & \textit{c10100}    & 2                               & 10               & 100              & 1419.000000                          & 1402 \\ 
			GRASP              & \textit{c10100}    & 3                               & 10               & 100              & 1419.000000                          & 1402 \\ 
			GRASP              & \textit{c10100}    & 4                               & 10               & 100              & 1419.000000                          & 1402 \\ 
			GRASP              & \textit{c10100}    & 5                               & 10               & 100              & 1419.000000                          & 1402 \\ 
			GRASP              & \textit{c10100}    & 6                               & 10               & 100              & 1419.000000                          & 1402 \\ 
			GRASP              & \textit{c10100}    & 7                               & 10               & 100              & 1419.000000                          & 1402 \\ 
			GRASP              & \textit{c10100}    & 8                               & 10               & 100              & 1419.000000                          & 1402 \\ 
			GRASP              & \textit{c10100}    & 9                               & 10               & 100              & 1419.000000                          & 1402 \\ \hline
			GRASP              & \textit{c20100}    & 0                               & 20               & 100              & 1268.000000                          & 1243 \\ 
			GRASP              & \textit{c20100}    & 1                               & 20               & 100              & 1268.000000                          & 1243 \\ 
			GRASP              & \textit{c20100}    & 2                               & 20               & 100              & 1268.000000                          & 1243 \\ 
			GRASP              & \textit{c20100}    & 3                               & 20               & 100              & 1268.000000                          & 1243 \\ 
			GRASP              & \textit{c20100}    & 4                               & 20               & 100              & 1268.000000                          & 1243 \\ 
			GRASP              & \textit{c20100}    & 5                               & 20               & 100              & 1268.000000                          & 1243 \\ 
			GRASP              & \textit{c20100}    & 6                               & 20               & 100              & 1268.000000                          & 1243 \\ 
			GRASP              & \textit{c20100}    & 7                               & 20               & 100              & 1268.000000                          & 1243 \\ 
			GRASP              & \textit{c20100}    & 8                               & 20               & 100              & 1268.000000                          & 1243 \\ 
			GRASP              & \textit{c20100}    & 9                               & 20               & 100              & 1268.000000                          & 1243 \\ \hline
			GRASP              & \textit{c05200}    & 0                               & 05               & 200              & 3490.000000                          & 3456 \\ 
			GRASP              & \textit{c05200}    & 1                               & 05               & 200              & 3490.000000                          & 3456 \\ 
			GRASP              & \textit{c05200}    & 2                               & 05               & 200              & 3490.000000                          & 3456 \\ 
			GRASP              & \textit{c05200}    & 3                               & 05               & 200              & 3490.000000                          & 3456 \\ 
			GRASP              & \textit{c05200}    & 4                               & 05               & 200              & 3490.000000                          & 3456 \\ 
			GRASP              & \textit{c05200}    & 5                               & 05               & 200              & 3490.000000                          & 3456 \\ 
			GRASP              & \textit{c05200}    & 6                               & 05               & 200              & 3490.000000                          & 3456 \\ 
			GRASP              & \textit{c05200}    & 7                               & 05               & 200              & 3490.000000                          & 3456 \\ 
			GRASP              & \textit{c05200}    & 8                               & 05               & 200              & 3490.000000                          & 3456 \\ 
			GRASP              & \textit{c05200}    & 9                               & 05               & 200              & 3490.000000                          & 3456 \\ \hline
			GRASP              & \textit{c10200}    & 0                               & 10               & 200              & 2860.000000                          & 2806 \\ 
			GRASP              & \textit{c10200}    & 1                               & 10               & 200              & 2860.000000                          & 2806 \\ 
			GRASP              & \textit{c10200}    & 2                               & 10               & 200              & 2860.000000                          & 2806 \\ 
			GRASP              & \textit{c10200}    & 3                               & 10               & 200              & 2860.000000                          & 2806 \\ 
			GRASP              & \textit{c10200}    & 4                               & 10               & 200              & 2860.000000                          & 2806 \\ 
			GRASP              & \textit{c10200}    & 5                               & 10               & 200              & 2860.000000                          & 2806 \\ 
			GRASP              & \textit{c10200}    & 6                               & 10               & 200              & 2860.000000                          & 2806 \\ 
			GRASP              & \textit{c10200}    & 7                               & 10               & 200              & 2860.000000                          & 2806 \\ 
			GRASP              & \textit{c10200}    & 8                               & 10               & 200              & 2860.000000                          & 2806 \\ 
			GRASP              & \textit{c10200}    & 9                               & 10               & 200              & 2860.000000                          & 2806 \\ \hline
			GRASP              & \textit{c20200}    & 0                               & 20               & 200              & 2457.000000                          & 2391 \\ 
			GRASP              & \textit{c20200}    & 1                               & 20               & 200              & 2457.000000                          & 2391 \\ 
			GRASP              & \textit{c20200}    & 2                               & 20               & 200              & 2457.000000                          & 2391 \\ 
			GRASP              & \textit{c20200}    & 3                               & 20               & 200              & 2457.000000                          & 2391 \\ 
			GRASP              & \textit{c20200}    & 4                               & 20               & 200              & 2457.000000                          & 2391 \\ 
			GRASP              & \textit{c20200}    & 5                               & 20               & 200              & 2457.000000                          & 2391 \\ 
			GRASP              & \textit{c20200}    & 6                               & 20               & 200              & 2457.000000                          & 2391 \\ 
			GRASP              & \textit{c20200}    & 7                               & 20               & 200              & 2457.000000                          & 2391 \\ 
			GRASP              & \textit{c20200}    & 8                               & 20               & 200              & 2457.000000                          & 2391 \\ 
			GRASP              & \textit{c20200}    & 9                               & 20               & 200              & 2457.000000                          & 2391 \\ \hline
			GAR SP             & \textit{e05100}    & 0                               & 05               & 100              & 13837.000000                          & 1267 \\ 
			GAR SP             & \textit{e05100}    & 1                               & 05               & 100              & 13837.000000                          & 1267 \\ 
			GAR SP             & \textit{e05100}    & 2                               & 05               & 100              & 13837.000000                          & 1267 \\ 
			GAR SP             & \textit{e05100}    & 3                               & 05               & 100              & 13837.000000                          & 1267 \\ 
			GAR SP             & \textit{e05100}    & 4                               & 05               & 100              & 13837.000000                          & 1267 \\ 
			GAR SP             & \textit{e05100}    & 5                               & 05               & 100              & 13837.000000                          & 1267 \\ 
			GAR SP             & \textit{e05100}    & 6                               & 05               & 100              & 13837.000000                          & 1267 \\ 
			GAR SP             & \textit{e05100}    & 7                               & 05               & 100              & 13837.000000                          & 1267 \\ 
			GAR SP             & \textit{e05100}    & 8                               & 05               & 100              & 13837.000000                          & 1267 \\ 
			GAR SP             & \textit{e05100}    & 9                               & 05               & 100              & 13837.000000                          & 1267 \\ \hline
			GAR SP             & \textit{e10100}    & 0                               & 10               & 100              & 12765.000000                          & 1156 \\ 
			GAR SP             & \textit{e10100}    & 1                               & 10               & 100              & 12765.000000                          & 1156 \\ 
			GAR SP             & \textit{e10100}    & 2                               & 10               & 100              & 12765.000000                          & 1156 \\ 
			GAR SP             & \textit{e10100}    & 3                               & 10               & 100              & 12765.000000                          & 1156 \\ 
			GAR SP             & \textit{e10100}    & 4                               & 10               & 100              & 12765.000000                          & 1156 \\ 
			GAR SP             & \textit{e10100}    & 5                               & 10               & 100              & 12765.000000                          & 1156 \\ 
			GAR SP             & \textit{e10100}    & 6                               & 10               & 100              & 12765.000000                          & 1156 \\ 
			GAR SP             & \textit{e10100}    & 7                               & 10               & 100              & 12765.000000                          & 1156 \\ 
			GAR SP             & \textit{e10100}    & 8                               & 10               & 100              & 12765.000000                          & 1156 \\ 
			GAR SP             & \textit{e10100}    & 9                               & 10               & 100              & 12765.000000                          & 1156 \\ \hline
			GRASP              & \textit{e20100}    & 0                               & 20               & 100              & 9164.000000                          & 8431 \\ 
			GRASP              & \textit{e20100}    & 1                               & 20               & 100              & 9164.000000                          & 8431 \\ 
			GRASP              & \textit{e20100}    & 2                               & 20               & 100              & 9164.000000                          & 8431 \\ 
			GRASP              & \textit{e20100}    & 3                               & 20               & 100              & 9164.000000                          & 8431 \\ 
			GRASP              & \textit{e20100}    & 4                               & 20               & 100              & 9164.000000                          & 8431 \\ 
			GRASP              & \textit{e20100}    & 5                               & 20               & 100              & 9164.000000                          & 8431 \\ 
			GRASP              & \textit{e20100}    & 6                               & 20               & 100              & 9164.000000                          & 8431 \\ 
			GRASP              & \textit{e20100}    & 7                               & 20               & 100              & 9164.000000                          & 8431 \\ 
			GRASP              & \textit{e20100}    & 8                               & 20               & 100              & 9164.000000                          & 8431 \\ 
			GRASP              & \textit{e20100}    & 9                               & 20               & 100              & 9164.000000                          & 8431 \\ \hline
			GAR SP             & \textit{e05200}    & 0                               & 05               & 200              & 29048.000000                          & 2492 \\ 
			GAR SP             & \textit{e05200}    & 1                               & 05               & 200              & 29048.000000                          & 2492 \\ 
			GAR SP             & \textit{e05200}    & 2                               & 05               & 200              & 29048.000000                          & 2492 \\ 
			GAR SP             & \textit{e05200}    & 3                               & 05               & 200              & 29048.000000                          & 2492 \\ 
			GAR SP             & \textit{e05200}    & 4                               & 05               & 200              & 29048.000000                          & 2492 \\ 
			GAR SP             & \textit{e05200}    & 5                               & 05               & 200              & 29048.000000                          & 2492 \\ 
			GAR SP             & \textit{e05200}    & 6                               & 05               & 200              & 29048.000000                          & 2492 \\ 
			GAR SP             & \textit{e05200}    & 7                               & 05               & 200              & 29048.000000                          & 2492 \\ 
			GAR SP             & \textit{e05200}    & 8                               & 05               & 200              & 29048.000000                          & 2492 \\ 
			GAR SP             & \textit{e05200}    & 9                               & 05               & 200              & 29048.000000                          & 2492 \\ \hline
			GAR SP             & \textit{e10200}    & 0                               & 10               & 200              & 26664.000000                          & 2330 \\ 
			GAR SP             & \textit{e10200}    & 1                               & 10               & 200              & 26664.000000                          & 2330 \\ 
			GAR SP             & \textit{e10200}    & 2                               & 10               & 200              & 26664.000000                          & 2330 \\ 
			GAR SP             & \textit{e10200}    & 3                               & 10               & 200              & 26664.000000                          & 2330 \\ 
			GAR SP             & \textit{e10200}    & 4                               & 10               & 200              & 26664.000000                          & 2330 \\ 
			GAR SP             & \textit{e10200}    & 5                               & 10               & 200              & 26664.000000                          & 2330 \\ 
			GAR SP             & \textit{e10200}    & 6                               & 10               & 200              & 26664.000000                          & 2330 \\ 
			GAR SP             & \textit{e10200}    & 7                               & 10               & 200              & 26664.000000                          & 2330 \\ 
			GAR SP             & \textit{e10200}    & 8                               & 10               & 200              & 26664.000000                          & 2330 \\ 
			GAR SP             & \textit{e10200}    & 9                               & 10               & 200              & 26664.000000                          & 2330 \\ \hline
			GAR SP             & \textit{e20200}    & 0                               & 20               & 200              & 25581.000000                          & 2237 \\ 
			GAR SP             & \textit{e20200}    & 1                               & 20               & 200              & 25581.000000                          & 2237 \\ 
			GAR SP             & \textit{e20200}    & 2                               & 20               & 200              & 25581.000000                          & 2237 \\ 
			GAR SP             & \textit{e20200}    & 3                               & 20               & 200              & 25581.000000                          & 2237 \\ 
			GAR SP             & \textit{e20200}    & 4                               & 20               & 200              & 25581.000000                          & 2237 \\ 
			GAR SP             & \textit{e20200}    & 5                               & 20               & 200              & 25581.000000                          & 2237 \\ 
			GAR SP             & \textit{e20200}    & 6                               & 20               & 200              & 25581.000000                          & 2237 \\ 
			GAR SP             & \textit{e20200}    & 7                               & 20               & 200              & 25581.000000                          & 2237 \\ 
			GAR SP             & \textit{e20200}    & 8                               & 20               & 200              & 25581.000000                          & 2237 \\ 
			GAR SP             & \textit{e20200}    & 9                               & 20               & 200              & 25581.000000                          & 2237 \\ \hline \hline
			
		\end{longtable}
	}

\end{document}
